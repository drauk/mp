% tex/conc/dgmpmax.tex   2018-1-13   Alan U. Kennington.
% $Id: tex/conc/dgmpmax.tex fb0adf70be 2017-11-24 12:18:46Z Alan U. Kennington $
% Some macros for use in the DG book and MetaPost files.
% General mathematics things should go in akmath.tex.
% The macros in _this_ file are mostly not standard enough for akmath.tex.
% In other words, most of these definitions are specific to my DG book.

% This file is needed because MetaPost doesn't inherit verbatimtex includes
% from included MetaPost macro files.
% This TeX macro file must be included in each mp file where it is needed.
% MetaPost also doesn't seem to like `%' comments in a verbatimtex block.

%------------------------------------------------------------------------------
% Prevent double loading:
\ifx\dgmpmaxtexloaded\relax\endinput\else\let\dgmpmaxtexloaded\relax\fi

% \emat: eqalign version.
\def\emat#1{$\displaystyle\eqalign{#1}$}

% \mmat: Matrix version.
\def\mmat#1{$\displaystyle\matrix{#1}$}

% \tmat: Plain text version. Remember "btex" starts in TeX horizontal mode.
% These are used only in metapost diagrams.
\def\tmat#1{% Single column per row, left justified.
 \vbox{\halign{##\hfil\cr#1\crcr}}}
\def\tmatB#1{% Two or more columns per row, all centred.
 \vbox{\halign{\hfil##\hfil&&\hfil##\hfil\cr#1\crcr}}}
\def\tmatLRSF#1#2#3#4#5{% Centred ten-point table.
 {\baselineskip#4\vbox{\halign{#2#5\vphantom{fy}###3\cr#1\crcr}}}}

\def\tmatseven#1{% Centred seven-point table.
 \tmatLRSF{#1}{\hfil}{\hfil}{9pt}{\sevenrm}}

\def\tmatten#1{% Centred ten-point table.
 \tmatLRSF{#1}{\hfil}{\hfil}{11pt}{\tenrm}}
\def\tmattenL#1{% Left justified ten-point table.
 \tmatLRSF{#1}{}{\hfil}{11pt}{\tenrm}}
\def\tmattenR#1{% Left justified ten-point table.
 \tmatLRSF{#1}{\hfil}{}{11pt}{\tenrm}}

% Special symbol for matrix for change of coordinate chart.
% \def\ccmat{Z}
\def\ccmat{J}

\def\opT{\overcirc T}       % Untagged tangent operators.
\def\tagT{\hat T}           % Tagged tangent operators.
\def\velT{\check T}         % Tagged tangent velocities.

% Ad-hoc notation for useless "tangent field covector" spaces.
\def\Thatbar{\skew3\hat{\bar T}}
% Ad-hoc \hat\bar notations for coordinates changes for mixed tensors.
\def\Bhatbar{\skew3\hat{\bar B}}
\def\ehatbar{\skew2\hat{\bar e}}
\def\Yhatbar{\hat{\bar Y}}

% Punctured closed ball.
\def\Bdotbar{\skew3\dot{\bar B}}

\def\opP{\overcirc P}       % Untagged tangent operator frames.
\def\tagP{\hat P}           % Tagged tangent operator frames.

% - - - - - - - - - - - - - - - - - - - - - - - - - - - - - - - - - - - - - - -
% More than 3 dots in the \dots macro.
{\catcode`@=11
% Four lower dots.
\gdef\ldotsIV{\mathinner{\ldotp\ldotp\ldotp\ldotp}}
\gdef\dotss{\relax\ifmmode\ldotsIV\else$\m@th\ldotsIV\,$\fi}
% Four centred dots.
\gdef\cdotsIV{\mathinner{\cdotp\cdotp\cdotp\cdotp}}
\gdef\cdotss{\relax\ifmmode\cdotsIV\else$\m@th\cdotsIV\,$\fi}
}

% - - - - - - - - - - - - - - - - - - - - - - - - - - - - - - - - - - - - - - -
\def\Xstyle{3}

\ifcase\Xstyle
% 0. Plain X definition.
\gdef\XX{X}                     % Space of cross-sections.
\gdef\XXo{\overcirc\XX}         % Space of local/partial cross-sections.
\gdef\XXhat{\hat\XX}            % De-tagged covariant form cross-sections?
\gdef\XXchk{\check\XX}          % De-tagged covariant form cross-sections?
\or
% 1. Squiggly X definition.
\gdef\XX{{\eusm X}}             % Space of cross-sections.
\gdef\XXo{\skew2\overcirc\XX}   % Space of local/partial cross-sections.
\gdef\XXhat{\skew{1.5}\hat\XX}  % De-tagged covariant form cross-sections?
\gdef\XXchk{\skew{1.5}\check\XX} % De-tagged covariant form cross-sections?
\or
% 2. Bold X definition.
\gdef\XX{{\bf X}}               % Space of cross-sections.
\gdef\XXo{\skew2\overcirc\XX}   % Space of local/partial cross-sections.
\gdef\XXhat{\skew{1.5}\hat\XX}  % De-tagged covariant form cross-sections?
\gdef\XXchk{\skew{1.5}\check\XX} % De-tagged covariant form cross-sections?
\or
% 3. Italic bold X definition.
\gdef\XX{{\cmmib X}}            % Space of cross-sections.
\gdef\XXo{\skew3\overcirc\XX}   % Space of local/partial cross-sections.
\gdef\XXhat{\skew{2.5}\hat\XX}  % De-tagged covariant form cross-sections?
\gdef\XXchk{\skew{2.5}\check\XX} % De-tagged covariant form cross-sections?
\fi

% 2017-7-2. Probably the \overlineright macro should be in akmath.tex.
% Could also define \overlineleftright and \overlineleft etc.?
\def\overlineright#1#2{%
 {\setbox0\hbox{$#2$}\hbox to\wd0{\hss$\overline{\hbox to#1{\hss$#2$}}$}}}
\def\XXoline{\overlineright{8pt}{\XX}{}}

\let\XXcut\XXoline              % Short-cut de-tagged form cross-sections.

% 2016-12-11. The notations \XXo and \XXloc need further study!
\def\XXloc{\XX_{\rm loc}}       % Local cross-sections with open domain.

% \gdef\vertXX{\,\bigr\vert\,}    % Restriction of space of cross-sections.
\gdef\vertXX{\,\vert\,}         % Restriction of space of cross-sections.

% - - - - - - - - - - - - - - - - - - - - - - - - - - - - - - - - - - - - - - -
\def\tagpartial{\hat\partial} % Tagged partial-symbol.
\def\tagD{\hat D}            % Tagged tangent operator for given tangent vector.

% - - - - - - - - - - - - - - - - - - - - - - - - - - - - - - - - - - - - - - -
% True and false propositions. (Tautologies and contradictions.)
% \def\propT{{\bf T}}
% \def\propF{{\bf F}}
\def\propT{\top}
\def\propF{\bot}
\def\valF{{\bf F}}
\def\valT{{\bf T}}
\def\valU{{\bf U}}

\def\valO{{\bf O}}
\def\valX{{\bf X}}

% \def\tval{\textop{Val}}
\def\tval{\mathop{\hbox{\bf t}}}    % This improves the vertical offset.
\def\tvalx{\textop{\bf\tau}}        % Note: There is no bold maths font!

% Bar-vee logic operator symbol (to be used in maths mode).
\def\barlor{\overline{\hbox to4.5pt{\hss$\vee$\hss}}}
\def\barlorW{\overline{\hbox to6.0pt{\hss$\vee$\hss}}}  % Wider bar.

% The but-not and not-but binary logical operators.
% \def\butnot{\mathbin{\hbox{\rlap{\kern2.5pt$|$}\hbox{\sybx\char"1B}}}}
% \def\notbut{\mathbin{\hbox{\rlap{\kern3.5pt$|$}\hbox{\sybx\char"1A}}}}
\def\butnot{%
 \mathbin{\hbox{\rlap{\kern2.3pt\sybx\char"6A}\hbox{\sybx\char"1B}}}}
\def\notbut{%
 \mathbin{\hbox{\rlap{\kern3.5pt\sybx\char"6A}\hbox{\sybx\char"1A}}}}

% Logic machine notation.
\def\LMput{\rightarrow}

%------------------------------------------------------------------------------
% Name map for a logical abstraction.
\def\Tnamemap{\mu}
% Name space for a logical abstraction.
\def\Tnamespace{{\cal N}}

% Sets of truth values.
\def\Tvalspace{\{\valF,\valT\}}
\def\TvalspaceU{\{\valF,\valT,\valU\}}

% Sets of exclusion values.
\def\Xvalspace{\{\valX,\valO\}}
\def\XvalspaceU{\{\valX,\valO,\valU\}}

% Map from concrete proposition space to truth values.
\def\Tvalmap{\tau}

% Concrete space for truth values.
\def\Ttruthspace{{\cal T}}
\def\Ttruthnamespace{\Tnamespace_\Ttruthspace}
\def\Ttruthnamemap{\Tnamemap_\Ttruthspace}

% Concrete space for a logical abstraction.
\def\Tpropspace{{\cal P}}
\def\Tpropnamespace{\Tnamespace_\Tpropspace}
\def\Tpropnamemap{\Tnamemap_\Tpropspace}

% Predicate calculus spaces: predicates, variables (objects), and object-maps.
\def\Tpredspace{{\cal Q}}
\def\Tprednamespace{\Tnamespace_\Tpredspace}
\def\Tprednamemap{\Tnamemap_\Tpredspace}

\def\Tvarspace{{\cal V}}
\def\Tvarnamespace{\Tnamespace_\Tvarspace}
\def\Tvarnamemap{\Tnamemap_\Tvarspace}

% Maps from objects to objects.
\def\Tfnspace{{\cal F}}
\def\Tfnnamespace{\Tnamespace_\Tfnspace}
\def\Tfnnamemap{\Tnamemap_\Tfnspace}

% Symbol for logical operations (truth functions).
\let\logop\theta

% Ad-hoc letters for spaces of operators and parentheses.
\def\opspace{{\cal O}}
\def\parenspace{{\cal B}}

% - - - - - - - - - - - - - - - - - - - - - - - - - - - - - - - - - - - - - - -
% Quoted math-mode text within the math context.
\def\hboxq#1{\hbox{``#1''}}
\def\hboxqm#1{\hbox{``$#1$''}}

% Quoted mathematical expression with extra space within text context.
\def\hboxqs#1{\hbox{``\ts#1\ts''}}
\def\hboxqsm#1{\hbox{``\ts$#1$\ts''}}

% - - - - - - - - - - - - - - - - - - - - - - - - - - - - - - - - - - - - - - -
% Substitution of logical expressions into a logical expression.
\def\Subs{\hbox{Subs}}

% List of dependencies at the end of a line of formal logical argument.
\def\bhyp#1{${}\dashv{}$(#1)}           % List of dependencies.
\def\bhypz{${}\dashv\emptyset$}         % Empty list of dependencies.

% Accents to indicate universal and existential variables.
\let\uhat\hat
\let\ehat\check

%------------------------------------------------------------------------------
% General construction of math symbol or text over an equality symbol.
\def\eqmath#1{\buildrel#1\over=}
\def\eqrm#1{\buildrel\rm#1\over=}
% Concrete equality relation.
\def\eqconc{\eqrm{c}}

% Wff name space and postfix/prefix spaces.
\def\WFF{{\rsfs W}}
\def\WFFpost{\WFF^-}
\def\WFFpre{\WFF^+}
\def\WFFin{\WFF^0}

\def\Tinterp{{\cal I}}
% \def\Tinterp{\textop{\hbox{\sybx I}}}
\def\Tnamemappost{\Tinterp^-}
\def\Tnamemappre{\Tinterp^+}
\def\Tnamemapin{\Tinterp^0}

% - - - - - - - - - - - - - - - - - - - - - - - - - - - - - - - - - - - - - - -
% The cardinality map and the "beth-cardinality" map.
\def\beth{\textop{beth}}
\def\card{\textop{card}}
\def\rank{\textop{rank}}
\def\betabar{\bar\beta}

% - - - - - - - - - - - - - - - - - - - - - - - - - - - - - - - - - - - - - - -
% Left and right element extraction operations for general ordered pairs.
\def\Left{\textop{Left}}
\def\Right{\textop{Right}}

% Real and imaginary parts of complex numbers.
% The customary gothic notations for real/imaginary parts are too ugly!
\def\Real{\textop{Re}}
\def\Imag{\textop{Im}}

%==============================================================================
% Tangent vectors and tangent operators.
\def\tang#1{t_{#1}}                     % Tangent vector.
\def\cotang#1{\tang{#1}^*}              % Tangent covector.
\def\tangtensor#1#2{\tang{#1}^{#2}}     % Tangent tensor.
\def\tangtwo#1{\tang{#1}^{[2]}}         % Second-order operator-like tangent.
\def\tangtang#1{\tang{#1}^{(2)}}        % Second-level tangent of a tangent.

\def\tangop#1{\partial_{#1}}
\def\tangoptwo#1{\partial_{#1}^{[2]}}   % Second-order operator.
\def\tangoptang#1{\partial_{#1}^{(2)}}  % Second-level tangent of tangent.

\def\tangvel#1{\check t_{#1}}           % Velocity vector.

% Left and right translation operators for tangent vectors.
\def\Lcirc{\overcirc L}
\def\Rcirc{\overcirc R}

% - - - - - - - - - - - - - - - - - - - - - - - - - - - - - - - - - - - - - - -
% Ricci curvature tensor, scalar curvature.
% \def\Ricci{\textord{Rc}}
\def\Ricci{\textord{\cal R}}
\def\scurv{\textord{\bf R}}

% - - - - - - - - - - - - - - - - - - - - - - - - - - - - - - - - - - - - - - -
% Matrix classes.
% \def\Msym#1{S_{#1}(\reals)} % Real symmetric matrices.
% \def\Msymnn#1{S_{#1}^+(\reals)} % Positive semi-def. real symmetric matrices.
% \def\Msymp#1{S_{#1}^{++}(\reals)} % Positive definite real symmetric matrices.
% \def\Msymnp#1{S_{#1}^-(\reals)} % Negative semi-def. real symmetric matrices.
% \def\Msymn#1{S_{#1}^{--}(\reals)} % Negative definite real symmetric matrices.

\def\MsymKs#1#2#3{\textop{Sym}#3({#1},{#2})} % Symmetric matrices template.
\def\MsymK#1#2{\MsymKs{#1}{#2}{}}        % Symmetric matrices.
\def\Msym#1{\MsymK{#1}{\reals}}          % Real symmetric matrices.
\def\Msymnn#1{\MsymKs{#1}{\reals}{^\szs{+}_0}} % Pos semi-def. symm matrices.
\def\Msymnp#1{\MsymKs{#1}{\reals}{^\szs{-}_0}} % Neg semi-def. symm matrices.
\def\Msymp#1{\MsymKs{#1}{\reals}{^\szs{+}}}    % Pos def. symmetric matrices.
\def\Msymn#1{\MsymKs{#1}{\reals}{^\szs{-}}}    % Pos def. symmetric matrices.

% Upper norm and lower norm of a matrix or linear map.
\def\lambdap{\lambda^\szs{+}}
\def\lambdan{\lambda^\szs{-}}
\def\mup{\mu^{+}}
\def\mun{\mu^{-}}

% Multi-dimensional matrices.
% \def\mmatrix{{\rsfs M}}
% \def\mmatrix{{\cal M}}
\def\mmatrix{{\cal A}}
% \def\mmatrix{{\rsfs A}}
% \def\mmatrix{{\eusm M}}

% - - - - - - - - - - - - - - - - - - - - - - - - - - - - - - - - - - - - - - -
% General ad hoc stuff local to dg.tex and conc.tex:
\def\Wedg#1{{\textstyle\bigwedge}^{\!#1}}   % Contravariant.
% \def\Wedg#1{{\textstyle\bigwedge}\!{}^{#1}}   % Contravariant.
\def\Lambb#1{\Lambda^{\!#1}}                % Contravariant with Lambda.
\def\Lamb#1{\Lambda_{#1}}                   % Covariant.
% \def\TLamb#1{{T\!\Lamb{#1}}}      % Tangent bundle of alternating tensors.
% \def\TWedg#1{{T\!\Wedg{#1}}}      % Tangent bundle of alternating tensors.
% \def\TLamb#1{\Lamb{#1}T}        % Tangent bundle of alternating tensors.
% \def\TWedg#1{\Wedg{#1}T}        % Tangent bundle of alternating tensors.
\def\wedgel{\mathop{\wedge}\limits}
\def\wedgedl{\mathop{\wedge}\displaylimits}
% \def\condep#1#2{\par\noindent\hangindent1cm #1\quad$\longrightarrow$\quad#2}
\def\condep#1#2{\par\noindent\hangindent1cm #2\quad$\longrightarrow$\quad#1}
\def\Topbar{\overline{\rm Top}}

% Triangle pointing down.
\mathchardef\triangledown="0235

% Textual functions and operators.
\def\Reg{\mathord{\cal K}}          % Other possibilities are F, S, X, Z.
\def\uls{\textop{UL}}               % Unrestricted linear space.
\def\fls{\textop{FL}}               % Free linear space.
\def\Fin{\textop{Fin}}              % Underlying set of a free linear space.
\def\FinConv{\textop{Fin}_0^1}      % Convex combinations in free linear space.

\def\Lip{\textop{Lip}}              % Lipschitz function space.
\def\Inc{\textop{Inc}}              % Space of increasing functions.
\def\Dec{\textop{Dec}}              % Space of decreasing functions.
\def\NonInc{\textop{NonInc}}        % Space of non-increasing functions.
\def\NonDec{\textop{NonDec}}        % Space of non-decreasing functions.
\def\Tree{\textop{Tree}}            % Space of trees.

\def\graph{\textop{graph}}          % Graph of a function.
% \def\loc{\textop{loc}}              % For integrability class subscripts.
\def\loc{\textord{loc}}             % For integrability class subscripts.

% \def\ddd{\hbox{\rsfs D}\,{}}      % A space of operators.
\def\ddd{{\rsfs D}\,{}}             % A space of operators.
\def\dbarf{\skew6\bar{\bar f}}      % Double bar over an f.
\def\dottimes{\mathbin{\dot\times}} % Common-domain function direct product.
\def\ddottimes{\mathbin{\ddot\times}} % Double-domain function direct product.

% Symbols for curves and paths.
\def\curves{{\rsfs C}}      % Curves.
\def\paths{{\rsfs P}}       % Paths.
% \def\paths{{\rsfs Q}}       % Paths.
% \def\path{\zeta}            % Generic symbol for path (equiv class of curves).
\def\path{Q}                % Generic symbol for path (equiv class of curves).

% Frame bundle.
\def\Frame{\textop{\cal F}} % Frame bundle. E.g. F_p^r(M).
% The set of all linearly independent $r$-tuples in R^n.
\def\fnr#1#2{F_{#1,#2}}
% Dual frame cross-section. (Coframe forms. Dual 1-forms.)
\let\dframe\sigma

% Velocity chart map for tangent spaces of Cartesian spaces.
% Possibilities: \xi, \kappa, \beta, \rho, etc.
\let\Vel\beta

% Lie groups.
\def\Adj{\textop{Adj}}      % Adjoint representation of Lie group.

% Symbols for connections and parallelism.
\let\konv\theta             % Connection on an OFB. T(M)->(E->T(E)).
\def\konz{{\bar\theta}}     % Connection on an OFB. E->(T(M)->T(E)).
\let\konV\beta              % Connection on an PFB. T(M)->(P->T(P)).
\def\konZ{{\bar\beta}}      % Connection on an PFB. P->(T(M)->T(P)).
\let\ppara\Theta            % Pathwise (topological) parallelism.

% Drop function and swap function for tangent bundle of a tangent bundle.
\let\dropf\varpi            % Drop function from vertical vectors to base space.
\let\swapf\Xi               % Swap function between identified vectors.

% Alternative quote style for Russian, German and French in the olden days.
% Requires the cyrillic font. The math versions come out too low.
% \def\llcyr{$\scriptscriptstyle\ll$}
% \def\ggcyr{$\scriptscriptstyle\gg$}
\def\llcyr{{\cyr<}}
\def\ggcyr{{\cyr>}}

% - - - - - - - - - - - - - - - - - - - - - - - - - - - - - - - - - - - - - - -
% Convex span (i.e. hull) of a set.
\def\convspan{\textop{conv}}

% Set of connected components of a subset of a topological space.
\def\ConnComp{\textop{ConnSet}}

% Total variation of a function.
\def\Var{\textop{Var}}

% Oscillation of a function.
\def\osc{\textord{osc}}

\def\On{\textord{On}}

% Levi-Civita alternating symbol. Could change this to \varepsilon some day?
\let\LCepsilon\epsilon

% Set of letterbox fits.
\def\lbfits{W}
% Letterbox envelope function.
\def\lbenv{\eta}

% Superscript perpendicular symbol to indicate a perpendicular spacing.
\def\perpsp#1{#1^\perp}
\def\perpspX#1{#1\rlap{${}^\perp$}}

% - - - - - - - - - - - - - - - - - - - - - - - - - - - - - - - - - - - - - - -
% Dini derivatives.
\def\Dur{\mathord{D^+_r}}
\def\Dlr{\mathord{D^-_r}}
\def\Dul{\mathord{D^+_\ell}}
\def\Dll{\mathord{D^-_\ell}}

% - - - - - - - - - - - - - - - - - - - - - - - - - - - - - - - - - - - - - - -
% The unit interval of a particular representation of the real numbers.
\def\Uint{\mathord{\bbm U}}
\def\Zbin{\mathord{\bbm V}}
% The floor/fraction representation of the real numbers.
\def\FFreals{\mathord{\bbm F}}

% - - - - - - - - - - - - - - - - - - - - - - - - - - - - - - - - - - - - - - -
% Ad-hoc notation for the connected component of an open set containing a point.
\def\Conn{\textop{Conn}}

% - - - - - - - - - - - - - - - - - - - - - - - - - - - - - - - - - - - - - - -
% Ad-hoc minus-spacing in \matrix for columns with \llap{$-{}$} before an entry.
\def\MinSp{\null\kern2pt}

% - - - - - - - - - - - - - - - - - - - - - - - - - - - - - - - - - - - - - - -
% Notation for the characteristic of a non-trivial unitary ring or field.
\def\Char{\textop{char}}

% - - - - - - - - - - - - - - - - - - - - - - - - - - - - - - - - - - - - - - -
% Rows and columns of matrices.
\def\Row{\textop{Row}}
\def\Col{\textop{Col}}
\def\Rowspan{\textop{Rowspan}}
\def\Colspan{\textop{Colspan}}
\def\Rowrank{\textop{Rowrank}}
\def\Colrank{\textop{Colrank}}

\def\nullity{\textop{nullity}}

% - - - - - - - - - - - - - - - - - - - - - - - - - - - - - - - - - - - - - - -
% Ad-hoc macro for the "slice" construction for Cartesian set-products.
\def\Slice{\textop{Slice}}
\def\Subst{\textop{Subst}}
\def\opLift{\textop{Lift}} % Not to be confused with \lift.

% - - - - - - - - - - - - - - - - - - - - - - - - - - - - - - - - - - - - - - -
% Constant vector field extensions of vectors on differentiable manifolds.
\def\Extn{\textop{Extn}}
