% tex/inputs/akmath.tex   2017-6-17   Alan U. Kennington.
% $Id: tex/inputs/akmath.tex cd758e15f1 2017-06-17 18:11:02Z Alan U. Kennington $
% Mathematical symbols. No general formatting macros.
% Just symbols and basic formula construction macros.
% Macros that are specific to the DG book should be in dgmpmax.tex.

%------------------------------------------------------------------------------
% Prevent double loading:
\ifx\akmathtexloaded\relax\endinput\else\let\akmathtexloaded\relax\fi

% Small caps and blackboard fonts:
% \font\tensmc        cmcsc10
\font\eightsmc      cmcsc8
% \font\tenbbb        msym10 scaled\magstephalf
% \font\tenbbb        msym10
% \def\smc{\tensmc}
% \def\Bbb{\tenbbb}

% Some sort of blackboard bold. Not the best one.
\font\tenbbb        bbold10
\def\bbb{\fam0\tenbbb}

% Fraktur font.
\font\tencmfrak     cmfrak
\let\cmfrak\tencmfrak

\font\tenplsy plsy10
\def\plsy{\tenplsy}

\font\seventcrm tcrm1000 at 7truept

% Magnified font to get decent medium-sized otimes symbol.
\font\bigtencmsy cmsy10 scaled 1315
\def\bigcmsy{\bigtencmsy}

%------------------------------------------------------------------------------
% Define bbm as a new family.
\newfam\bbmfam
\font\tenbbm        bbm10
\font\ninebbm       bbm9
\font\eightbbm      bbm8
\font\sevenbbm      bbm7
\font\fivebbm       bbm5
\textfont\bbmfam=\tenbbm
\scriptfont\bbmfam=\sevenbbm
\scriptscriptfont\bbmfam=\fivebbm
% The \fam is for maths-mode. The \tenbbm is for horiz-mode. Grrr.....
\def\bbm{\fam\bbmfam\tenbbm}

\def\bbmeight{% A very fake font family! Do not use unless very desperate!!!!
 \textfont0=\eightbbm%
 \scriptfont0=\fivebbm%
 \scriptscriptfont0=\fivebbm%
 \textfont1=\eightbbm%
 \scriptfont1=\fivebbm%
 \scriptscriptfont1=\fivebbm%
 \textfont\bbmfam=\eightbbm%
 \scriptfont\bbmfam=\fivebbm%
 \scriptscriptfont\bbmfam=\fivebbm%
 \fam\bbmfam%
 \eightbbm%
% Z% Test.
 }

%------------------------------------------------------------------------------
% Define rsfs as a new family.
\newfam\rsfsfam
\font\tenrsfs       rsfs10
\font\sevenrsfs     rsfs7
\font\fiversfs      rsfs5
\textfont\rsfsfam=\tenrsfs
\scriptfont\rsfsfam=\sevenrsfs
\scriptscriptfont\rsfsfam=\fiversfs
% The \fam is for maths-mode. The \tenrsfs is for horiz-mode. Grrr.....
\def\rsfs{\fam\rsfsfam\tenrsfs}

%------------------------------------------------------------------------------
% Define euxm as a new family.
% \newfam\euxmfam
% \font\teneuxm       euxm10
% \font\seveneuxm     euxm7
% \font\fiveeuxm      euxm5
% \textfont\euxmfam=\teneuxm
% \scriptfont\euxmfam=\seveneuxm
% \scriptscriptfont\euxmfam=\fiveeuxm
% % The \fam is for maths-mode. The \teneuxm is for horiz-mode. Grrr.....
% \def\euxm{\fam\euxmfam\teneuxm}

%------------------------------------------------------------------------------
% Define eusm as a new family.
\newfam\eusmfam
\font\teneusm       eusm10
\font\seveneusm     eusm7
\font\fiveeusm      eusm5
\textfont\eusmfam=\teneusm
\scriptfont\eusmfam=\seveneusm
\scriptscriptfont\eusmfam=\fiveeusm
% The \fam is for maths-mode. The \teneusm is for horiz-mode. Grrr.....
\def\eusm{\fam\eusmfam\teneusm}

%------------------------------------------------------------------------------
% Define cmmib as a new family.
\newfam\cmmibfam
\font\tencmmib       cmmib10
\font\sevencmmib     cmmib7
\font\fivecmmib      cmmib5
\textfont\cmmibfam=\tencmmib
\scriptfont\cmmibfam=\sevencmmib
\scriptscriptfont\cmmibfam=\fivecmmib
% The \fam is for maths-mode. The \tencmmib is for horiz-mode. Grrr.....
\def\cmmib{\fam\cmmibfam\tencmmib}

%------------------------------------------------------------------------------
% Stuff to do underlining: a line under the base of an hbox, not under the box!
% Should be in a separate macro file.
\newdimen\ulinedepth \ulinedepth3pt %
\def\uline#1{{\vtop{\hbox{#1}\skip0 \prevdepth\vskip-\skip0 %
 \vskip\ulinedepth\hrule\vskip-\ulinedepth\vskip\skip0}}}

% Put an overline over something at a fixed height.
% There is apparently no \prevheight. Also, a \vbox is processed top-down.
% This height is chosen for inf, sup, min and max.
\newdimen\olineheight \olineheight=6.5pt %
\def\oline{\relax % \relax, in case this comes first in \halign
  \ifmmode\def\next{\mathpalette\matholine}\else\let\next\makeoline%
  \fi\next}
\def\makeoline#1{\setbox0\hbox{#1}\finoline}
\def\matholine#1#2{\setbox0\hbox{$\mathsurround0pt#1{#2}$}\finoline}
\def\finoline{\ht0=\olineheight\overline{\hbox{\box0}}}

%------------------------------------------------------------------------------
% Centered horizontal smash/squash.
\def\clap#1{\hbox to0pt{\hss#1\hss}}
\def\clapS#1#2{\hbox to#1{\hss#2\hss}}

% Scripts which should lie outside a faction, e.g. in dx^k/dt.
\def\rlapsup#1{\rlap{${}^{#1}$}}
\def\rlapsub#1{\rlap{${}_{#1}$}}

% Lap-macros for maths mode in script style.
\def\rlaps#1{\rlap{$\scriptstyle#1$}}
\def\llaps#1{\llap{$\scriptstyle#1$}}
\def\claps#1{\clap{$\scriptstyle#1$}}

%------------------------------------------------------------------------------
\def\overcirc{\mathaccent"7017 }

% Textual function names.
\def\textop#1{\mathop{\rm#1}\nolimits}
\def\textopdl#1{\mathop{\rm#1}}         % Default mathop is \displaylimits.
\def\textord#1{\mathord{\rm#1}}
\def\textbin#1{\mathbin{\rm#1}}
\def\textrel#1{\mathrel{\rm#1}}

% Vertically balanced version of \textop.
\def\opstrut#1{\mathop{\rm\strut#1}}

% Wider than normal overbar:
% \def\Bar#1{\vbox{\offinterlineskip\ialign{##\crcr%
%  \rm\hidewidth\char"7B\hidewidth\cr\noalign{\vskip-0.25ex}$#1$\cr}}}
\def\Barx#1#2{% #1 = text, #2 = vertical distance.
 \vbox{\offinterlineskip\ialign{##\crcr\hidewidth\setbox0=\hbox{\rm\char"7B}%
 \ht0=#2 \box0\hidewidth\cr\noalign{\vskip-#2}$#1$\cr}}}
\def\Bar#1{%
 \vbox{\offinterlineskip\ialign{##\crcr\hidewidth\setbox0=\hbox{\rm\char"7B}%
 \ht0=0.25ex \box0\hidewidth\cr\noalign{\vskip-0.25ex}$#1$\cr}}}
\def\Barmin#1{% Use a minus sign. I have no idea what I'm doing here!
 \vbox{\offinterlineskip\ialign{##\crcr%
 \hidewidth\setbox0=\hbox{${}-{}$}%
% \dimen0=\ht0\showthe\dimen0%
 \ht0=0.5pt \ht0=0pt \box0\hidewidth\cr%
 \noalign{\vskip-2.0pt}%
 $#1$\cr}}}
% \def\Ombar{\mathchoice{\Bar\Omega}{\Bar\Omega}{\bar\Omega}{\bar\Omega}}
\def\Ombar{\mathchoice{\Barmin\Omega}{\Barmin\Omega}{\bar\Omega}{\bar\Omega}}

% A hopefully thicker overbar which varies with \mathchoice.
\def\hzBarstyle#1#2{%
 {\def\hzbarsty{\ifcase#2\displaystyle\or\textstyle\or\scriptstyle%
  \else\scriptscriptstyle\fi}%
% \def\hzbar{\ifcase#2 D\or T\or S\else Z\fi}%
 \setbox0=\hbox{$\hzbarsty#1$}\dimen0\wd0\advance\dimen0-2.5pt%
 \vbox{\mathsurround0pt\offinterlineskip\ialign{##\crcr%
  \hidewidth\setbox0=\hbox{\hsize\dimen0\vbox{\hrule height1pt}}\ht0=0.25ex%
  \box0\hidewidth\cr%
  \noalign{\vskip-0.25ex}%
  $\hzbarsty#1$\cr}}%
 }}
\def\hzBar#1{% There should be a built-in facility in TeX to do this!
 \mathchoice{\hzBarstyle{#1}{0}}{\hzBarstyle{#1}{1}}%
  {\hzBarstyle{#1}{2}}{\hzBarstyle{#1}{3}}%
 }

%------------------------------------------------------------------------------
% Superscript/subscript sign to balance with zero in other position.
% This should only be used when the main character has some spare space
% at the top right so that the sign can be left-shifted a little.
% (This macro is distantly related to the \crfldg macro.)
\def\szs#1{%
 {% Set box 0 to an hbox containing a script-style "0".
  \setbox0\hbox{$\scriptstyle0$}%
  % Set dimen 0 to the width of box 0.
  \dimen0\wd0 %
  % Set box 0 to an hbox containing the parameter character in script style.
  \setbox0\hbox{$\scriptstyle#1$}%
  % Subtract the parameter-character width from the 0 width.
  \advance\dimen0-\wd0 %
  % Set dimen 0 to (width('0') - width(#1))/2.
  % Note that it is expected that dimen 0 will be negative for #1 = '+' or '-'.
  \divide\dimen0 by2 %
  % Move #1 to the right so that it is centred within the width of a '0'.
  \kern\dimen0{#1}%
 }}

%------------------------------------------------------------------------------
% Sets of ordinal numbers.
\ifx01
\def\om{\omega}         % Not used much probably.
\def\omp{\omega^+}      % Not used much probably.
\def\omcomp#1{\om\setminus\{#1\}}
\def\ompcomp#1{\omp\setminus\{#1\}}

\def\oldintnotation{%
% \def\ints{\hbox{\Bbb Z}}
% \def\ints{\mathord{\bf Z}}%
 \def\ints{{\bbm Z}}%
 \def\nnints{{\ints^+}}%
% \def\xnnints{{\nnints\bcup\{\infty\}}}%
 \def\xnnints{{\omega^+}}%
 \def\nnintsX{{\nnints\setbox0\hbox{$\scriptstyle{}^+$}\hskip-\wd0}}%
% \def\posints{\nnints\setminus\{0\}}%
% \def\posints{\mathord{\bf N}}%
% \def\posints{{\mathord{{\rm I\kern-1.5pt N}}}}%
 \def\posints{{\bbm N}}%
 \def\xposints{{\bbm N}\cup\{\infty\}}%
 }
\def\newintnotation{%
 % \def\ints{\hbox{\Bbb Z}}
 \def\ints{\mathord{\bf Z}}%
 \let\nnints\om%
 \let\xnnints\omp%
 \let\nnintsX\nnints%
 \def\posints{\omcomp{0}}%
 }
% \oldintnotation
\fi

% - - - - - - - - - - - - - - - - - - - - - - - - - - - - - - - - - - - - - - -
% Some ordinal number things.
\def\Ord{\textop{Ord}}

% - - - - - - - - - - - - - - - - - - - - - - - - - - - - - - - - - - - - - - -
% Sets of integers.
% \def\ints{\hbox{\Bbb Z}}
% \def\ints{\mathord{\bf Z}}%
\def\ints{\mathord{\bbm Z}}%
\def\intsgt{\ints^+}
\def\intsge{\ints_0^+}
\def\intslt{\ints^-}
\def\intsle{\ints_0^-}

% Warning: The intsx macros seem to create an oversize Z when the macro
% appears in a subscript or superscript. Must fix this!!
% \def\intsx{\ints^*}
% \def\intsx{\Bar\ints}
\def\intsx{\Barmin\ints}
% \def\intsxgt{\intsgt\cup\{\infty\}}
% \def\intsxge{\intsge\cup\{\infty\}}
\def\intsxgt{\intsx^+}
\def\intsxge{\intsx^+_0}
\def\intsxlt{\intsx^-}
\def\intsxle{\intsx^-_0}

% Warning: The macro intsgeX looks completely wrong to me. Don't use it!
% \def\intsgeX{{\intsge\setbox0\hbox{$\scriptstyle{}_0^+$}\hskip-\wd0}}

% Natural numbers.
\def\intsnat{\mathord{\bbm N}}
% \def\intsnatx{\intsnat\cup\{\infty\}}
% \def\intsnatx{\Bar\intsnat}
\def\intsnatx{\Barmin\intsnat}

% Obsolete definitions.
% \def\nnints{{\ints^+}}%
% % \def\xnnints{{\nnints\bcup\{\infty\}}}%
% \def\xnnints{{\omega^+}}%
% \def\nnintsX{{\nnints\setbox0\hbox{$\scriptstyle{}^+$}\hskip-\wd0}}%
% % \def\posints{\nnints\setminus\{0\}}%
% % \def\posints{\mathord{\bf N}}%
% % \def\posints{{\mathord{{\rm I\kern-1.5pt N}}}}%
% \def\posints{{\bbm N}}%
% \def\xposints{{\bbm N}\cup\{\infty\}}%

% - - - - - - - - - - - - - - - - - - - - - - - - - - - - - - - - - - - - - - -
% Rational numbers.
\def\rats{\mathord{\bbm Q}}
\def\ratsgt{\rats^\szs{+}}
\def\ratsge{\rats_0^\szs{+}}
\def\ratslt{\rats^\szs{-}}
\def\ratsle{\rats_0^\szs{-}}

% \def\ratsx{\Bar\rats}
\def\ratsx{\Barmin\rats}
\def\ratsxgt{\ratsx^\szs{+}}
\def\ratsxge{\ratsx_0^\szs{+}}
\def\ratsxlt{\ratsx^\szs{-}}
\def\ratsxle{\ratsx_0^\szs{-}}

% Obsolete.
% \def\oldrats{\mathord{Q}}
% \def\nnrats{\rats_0^+}
% \def\posrats{\rats^+}

% - - - - - - - - - - - - - - - - - - - - - - - - - - - - - - - - - - - - - - -
% Real numbers.
% \def\reals{{\mathord{\hbox{\Bbb R}}}}
% \def\reals{{\mathord{{\rm I\kern-1.5pt R}}}} % Fake blackboard bold R.
\def\reals{\mathord{\bbm R}}
\def\realseight{\mathord{\hbox{\bbmeight R}}}
\def\realsgt{\reals^\szs{+}}
\def\realsge{\reals^\szs{+}_0}
\def\realslt{\reals^\szs{-}}
\def\realsle{\reals^\szs{-}_0}

% \def\realsx{\reals^*}
% \def\realsxge{\realsge\cup\{\infty\}}
% \def\realsx{\Bar\reals}
\def\realsx{\Barmin\reals}
% \def\realsx{\overline\reals}      % This overline is too thick and too wide.
\def\realsxgt{\realsx^\szs{+}}
\def\realsxge{\realsx^\szs{+}_0}
\def\realsxlt{\realsx^\szs{-}}
\def\realsxle{\realsx^\szs{-}_0}

% Obsolete real number notations.
% \def\xreals{\reals^*}
% \def\nnreals{\reals^+} % Obsolete.
% \def\xnnreals{\reals^+\cup\{\infty\}} % Obsolete.
% % \def\nnreals{\reals_0^+}
% % \def\posreals{\reals^+}

% - - - - - - - - - - - - - - - - - - - - - - - - - - - - - - - - - - - - - - -
% Complex numbers.
% \def\oldcomplex{{\mathord{{\rm \hbox{\vrule}\kern-1.5pt C}}}}
\def\complex{\mathord{\bbm C}}

%------------------------------------------------------------------------------
% Lebesgue measure. Unfortunately, I use \rsfs L for multilinear spaces already.
\def\leb{\mathord{\bbm L}}

% Analytical limits.
\newdimen\limpluswidth{% Width of ${}^+$ for subscripts of limits.
 \setbox0\hbox{$\scriptstyle{}^+$}\global\limpluswidth\wd0 }
\def\limplus{\rlap{$\scriptstyle{}^+$}}
\def\limplusskip{\hskip\limpluswidth}

% Harpoon arrow on top of given text.
% Should have generic macros for putting various arrows on top of things.
{\catcode`\@=11 % Include @ in macro names.
\gdef\rightharpoonupfill{$\m@th\smash-\mkern-7mu%
 \cleaders\hbox{$\mkern-2mu\smash-\mkern-2mu$}\hfill
 \mkern-7mu\mathord\rightharpoonup$}
\gdef\overrightharpoonup#1{\vbox{\m@th\ialign{##\crcr
 \rightharpoonupfill\crcr\noalign{\kern-\p@\nointerlineskip}
 $\hfil\displaystyle{#1}\hfil$\crcr}}}
} % End of catcode scope.

% - - - - - - - - - - - - - - - - - - - - - - - - - - - - - - - - - - - - - - -
% Dot on equals for abbreviations of specification tuples.
\def\rightfanD{\mathrel{\hbox{\rlap{$-$}$<$}}}
\let\rightfanT\rightfanD
\def\rightfanS{\mathrel{\hbox{\rlap{$\scriptstyle-$}$\scriptstyle<$}}}
\def\rightfanSS{%
 \mathrel{\hbox{\rlap{$\scriptscriptstyle-$}$\scriptscriptstyle<$}}}
\def\rightfan{\mathchoice{\rightfanD}{\rightfanT}{\rightfanS}{\rightfanSS}}

\def\leftfanD{\mathrel{\hbox{\rlap{$-$}$>$}}}
\let\leftfanT\leftfanD
\def\leftfanS{\mathrel{\hbox{\rlap{$\scriptstyle-$}$\scriptstyle>$}}}
\def\leftfanSS{%
 \mathrel{\hbox{\rlap{$\scriptscriptstyle-$}$\scriptscriptstyle>$}}}
\def\leftfan{\mathchoice{\leftfanD}{\leftfanT}{\leftfanS}{\leftfanSS}}

\let\abbrevf\rightfan     % Abbreviation for/of/forward. X\abbrevf(X,T).
\let\abbrevb\leftfan      % Abbreviation by/backward. (X,T)\abbrevf X.

\let\ora\overrightarrow
\let\orharp\overrightharpoonup

% - - - - - - - - - - - - - - - - - - - - - - - - - - - - - - - - - - - - - - -
% Fixed spacing for various operators.
\def\muskA{\mskip3mu}
\def\muskB{\mskip4mu}
\def\muskC{\mskip5mu}

% \def\assert{\mathbin{\vdash}}
% \def\assert{\muskC\vdash\muskB}
% \def\assert{\mathrel{\vdash}}
\let\assert\vdash                       % Relation atom.
\def\asserts{\muskC\vdash\muskB}        % Extra space within an assertion.

% Assertion in a particular propositional calculus.
% \def\assertX#1{\muskC\vdash_{#1}\muskB}
\def\assertX#1{\assert_{#1}}
\def\assertsX#1{\muskC\assert_{#1}}

% Two-way assertion.
\def\assertt{\mathrel{\dashv\kern1.4pt\vdash}} % Bidirectional assertion.
\def\assertts{\muskC\assertt\muskC}            % Extra space.

\def\asserttX#1{\assertt_{#1}}
\def\asserttsX#1{\muskC\assertt_{#1}}

% - - - - - - - - - - - - - - - - - - - - - - - - - - - - - - - - - - - - - - -
% Conditional operators.
\def\imprel{\mathrel{\Rightarrow}}          % Implies.
\def\impshort{\muskA\Rightarrow\muskA}
\def\implong{\muskC\Rightarrow\muskC}
\def\impquad{\quad\Rightarrow\quad}
\let\implies\imprel

\def\notimprel{\mathrel{\not\Rightarrow}}   % Not implies.
\def\notimpshort{\muskA\not\Rightarrow\muskA}
\def\notimplong{\muskC\not\Rightarrow\muskC}
\def\notimpquad{\quad\not\Rightarrow\quad}
\let\notimplies\notimprel

\def\impbyrel{\mathrel{\Leftarrow}}         % Implied by.
\def\impbyshort{\muskA\Leftarrow\muskA}
\def\impbylong{\muskC\Leftarrow\muskC}
\def\impbyquad{\quad\Leftarrow\quad}
\let\impliedby\impbyrel

\def\iffrel{\mathrel{\Leftrightarrow}}      % If and only if.
\def\iffshort{\muskA\Leftrightarrow\muskA}
\def\ifflong{\muskC\Leftrightarrow\muskC}
\def\iffquad{\quad\Leftrightarrow\quad}

% \def\fspace{{\cal F}}
% \def\powerset{{\cal P}}
% \def\powerset{{\mathord{{\rm I\kern-1.5pt P}}}}
\def\powerset{\mathord{\bbm P}}

% List spaces.
% \def\List{{\cal L}} % List space.
\def\List{\mathop{\rm List}\nolimits}
% \def\barList{\mathop{\rm List}\nolimits}
% \def\barList{\mathop{\Bar{\rm List}}\nolimits}
\def\barList{\mathop{\overline{\rm List}}\nolimits}

% Multinomial coefficient spaces.
\def\Mult{\mathop{\rm Mult}\nolimits}
\def\barMult{\mathop{\overline{\rm Mult}}\nolimits}

% Notations for maps for sets of bounds. These look very uneven and clumsy.
% \def\infbar{\mathop{\overline{\rm inf}}}
% \def\supbar{\mathop{\overline{\rm sup}}}
% \def\minbar{\mathop{\overline{\rm min}}}
% \def\maxbar{\mathop{\overline{\rm max}}}

% Notations for maps for sets of bounds.
% Must be used in math mode.
\def\infbar{\mathop{\oline{\rm inf}}}
\def\supbar{\mathop{\oline{\rm sup}}}
\def\minbar{\mathop{\oline{\rm min}}}
\def\maxbar{\mathop{\oline{\rm max}}}
\def\lbbar{\mathop{\oline{\rm lb}}}
\def\ubbar{\mathop{\oline{\rm ub}}}

% These X-versions are intended to be used with a superscript.
% Must be used in math mode.
\def\infbarSR#1{{}^{\raise1pt\hbox{$\scriptstyle#1$}}}
\def\infbarS#1{{\oline{\rm inf}}\infbarSR{#1}}
\def\supbarS#1{{\oline{\rm sup}}\infbarSR{#1}}
\def\minbarS#1{{\oline{\rm min}}\infbarSR{#1}}
\def\maxbarS#1{{\oline{\rm max}}\infbarSR{#1}}

\def\starList{\List^*}
\def\Listx{\barList} % Could be barList or starList.

\def\lift{\mathop{\rm lift}\nolimits} % Not to be confused with \opLift.

\def\mcap{\mathop{\textstyle\bigcap}}
\def\mcup{\mathop{\textstyle\bigcup}}
\def\smcap{\mathop\cap\limits}
\def\smcup{\mathop\cup\limits}
% \def\bcap{\mathbin{\textstyle\cap}} % Binary \cap.
% \def\bcup{\mathbin{\textstyle\cup}} % Binary \cup.
\def\bcap{\mathbin{\cap}} % Binary \cap.
\def\bcup{\mathbin{\cup}} % Binary \cup.

\def\mprod{\mathop{\prod}}   % Forces sub/superscripts under/above symbol.

\def\mtimes{\mathop{\times}}   % Forces sub/superscripts under/above symbol.
\def\mcirctimes{% Forces sub/superscripts under/above symbol.
 \mathop{\overcirc\times}}
\def\lowtimes#1{\times_{\lower1pt\hbox{\kern-1pt$\scriptstyle#1$}}}

\def\motimes{\mathop{\textstyle\bigotimes}}
\def\smotimes{\mathop{\textstyle\otimes}}
\def\sotimes{{\textstyle\bigotimes}} % Forces sub/superscripts to the side.
\def\lowotimes#1{\otimes_{\lower1pt\hbox{$\scriptstyle#1$}}}

\def\moplus{\mathop{\textstyle\bigoplus}}
\def\smoplus{\mathop{\oplus}}
\def\soplus{{\textstyle\bigoplus}}

% - - - - - - - - - - - - - - - - - - - - - - - - - - - - - - - - - - - - - - -
% Some symbols which plain TeX doesn't give nice results for.
% Small \times.
\let\timesb\times                           % Binary
\def\timesc{\mathop\times\displaylimits}    % Centred.
\def\timesr{\mathop\times\nolimits}         % Right-hand side.

% Small \otimes.
\def\otimesb{\,\mathord\otimes\,}           % Small, binary.
\def\otimesc{\mathop\otimes\displaylimits}  % Small, centred.
\def\otimesr{\mathop\otimes\nolimits}       % Small, right-hand side.

% Big \otimes.
\def\botimes{\hbox{\bigcmsy\char"0A}}       % Large, solitary.
\def\botimesb{\,\botimes\,}                 % Large, binary.
\def\botimesc{\mathop{\botimes}\displaylimits} % Large, centred.
\def\botimesr{\mathop{\botimes}\nolimits}   % Large, RHS.

% Small \oplus.
\def\oplusb{\,\mathord\oplus\,}             % Small, binary.
\def\oplusc{\mathop{\textstyle\oplus}\displaylimits} % Small, centred.
\def\oplusr{\mathop\oplus\nolimits}         % Small, right-hand side.

% Big \oplus.
\def\boplus{\hbox{\bigcmsy\char"08}}        % Large, solitary.
\def\boplusb{\,\boplus\,}                   % Large, binary.
\def\boplusc{\mathop{\boplus}\displaylimits} % Large, centred.
\def\boplusr{\mathop{\boplus}\nolimits}     % Large, RHS.

% - - - - - - - - - - - - - - - - - - - - - - - - - - - - - - - - - - - - - - -
% Circles over arrows.
% \def\arrowcircbig{\mathop{\to}\limits^\circ}
\def\arrowcircbig{\mathrel{\mathop{\to}\limits^\circ}}
\def\arrowcircsmall{\mathrel{\overcirc\to}}
% \let\arrowcirc\arrowcircbig
\let\arrowcirc\arrowcircsmall
% width(\rightarrow) = 10pt, width(\longrightarrow) = 16pt, width(\circ) = 5pt.
% \bgroup\setbox0\hbox{$\rightarrow$}\dimen0\wd0 \divide\dimen0 by2

% William Tell symbols.
\gdef\longwtell{%
 \mathrel{\hbox{\hbox to0pt{\kern5.5pt$\circ$\hss}$\longrightarrow$}}}
\gdef\medwtell{%
 \mathrel{\hbox to13pt{\hbox to0pt{\kern4pt$\circ$\hss}\rightarrowfill}}}
\gdef\smallwtell{\mathrel{\hbox{\hbox to0pt{\kern2pt$\circ$\hss}$\rightarrow$}}}
\let\wtell\medwtell

\def\smallsum{\mathop{\textstyle\sum}}
\def\smallprod{\mathop{\textstyle\prod}}
\let\forsome\exists         % Correct meaning of the symbol.
\def\suchthat{;\muskA}      % Creates space in set definitions.

\def\relcompact{\mathrel{\subset\subset}}    % Relatively compact relation.
\def\subsetneqsym{%
 \lower1.5pt\vbox{\offinterlineskip\lineskip-1pt\halign{%
 \hfil$\scriptstyle##$\hfil\cr\subset\cr\neq\cr}}}
\def\supsetneqsym{%
 \lower1.5pt\vbox{\offinterlineskip\lineskip-1.4pt\halign{%
 \hfil$\scriptstyle##$\hfil\cr%
 \supset\cr=\llap{$\scriptstyle\backslash$\kern1.4pt}\cr}}}
\def\subsetneq{\mathrel{\subsetneqsym}}
\def\supsetneq{\mathrel{\supsetneqsym}}

% - - - - - - - - - - - - - - - - - - - - - - - - - - - - - - - - - - - - - - -
\def\text#1{\hbox{\rm#1}}
\def\textsp#1{\hbox{\rm\space#1\space}}
\def\textspl#1{\hbox{\rm\space#1}}
\def\textspr#1{\hbox{\rm#1\space}}
\def\textquad#1{\hbox{\rm\quad#1\quad}}
\def\textSP[#1]#2{\hbox{\rm\hskip#1\relax#2\hskip#1\relax}}
\def\textS#1{\hbox{\sevenrm#1}}

\let\gp\text            % Classical Lie group.
\let\gpS\textS          % Classical Lie group in a subscript or superscript.
% \let\gpl\uline          % Lie algebra of a classical Lie group.
\def\gpl#1{\uline{#1}}  % Lie algebra of a classical Lie group.
\def\ts{\ifmmode\,\else\thinspace\fi}
\def\tsland{\ts\land\ts}
\def\tslor{\ts\lor\ts}

% \cdot needs more space around it when used as function parameter place-holder.
\def\cdotsp{\,\cdot\,}

% - - - - - - - - - - - - - - - - - - - - - - - - - - - - - - - - - - - - - - -
% Physical units:
\def\tsx#1{\thinspace\text{#1}}

\def\nS{\tsx{nS}}
\def\uS{\tsx{$\mu$S}}
\def\mS{\tsx{mS}}
\def\cS{\tsx{cS}}
\def\Sec{\tsx{S}}

\def\Gbps{\tsx{Gbps}}
\def\Mbps{\tsx{Mbps}}
\def\kbps{\tsx{kbps}}
\def\bps{\tsx{bps}}

\def\GHz{\tsx{GHz}}
\def\MHz{\tsx{MHz}}
\def\kHz{\tsx{kHz}}
\def\Hz{\tsx{Hz}}

\def\GB{\tsx{GB}}
\def\MB{\tsx{MB}}
\def\kB{\tsx{kB}}
\def\frac#1#2{{#1\over#2}}
\def\pfrac#1#2{{\partial#1\over\partial#2}}
\def\tfrac#1#2{{\textstyle{{#1}\over{#2}}}}
\def\dfrac#1#2{{\displaystyle{{#1}\over{#2}}}}

% Copied from TeX book, page 311.
\def\slashfrac#1/#2{%
 \leavevmode\kern.1em\raise.5ex\hbox{\the\scriptfont0 #1}%
 \kern-0.1em/\kern-0.15em\lower.25ex\hbox{\the\scriptfont0 #2}}

% \def\dl#1#2{\delta^{#1}_{#2}}
% \def\dl#1#2{\delta_{#1#2}}
\def\dl#1#2{\delta_{#1}^{#2}}
\def\Dl#1#2{\delta^{#1}_{#2}}

% - - - - - - - - - - - - - - - - - - - - - - - - - - - - - - - - - - - - - - -
\def\csmash#1{\hbox to0pt{\hss#1\hss}}

% Top-only and bottom-only versions of \smash. See plain.tex.
\def\tsmash{\relax % \relax, in case this comes first in \halign
  \ifmmode\def\next{\mathpalette\mathtsmash}\else\let\next\maketsmash%
  \fi\next}
\def\maketsmash#1{\setbox0\hbox{#1}\fintsmash}
\def\mathtsmash#1#2{\setbox0\hbox{$\mathsurround0pt#1{#2}$}\fintsmash}
\def\fintsmash{\ht0=0pt \box0}

\def\bsmash{\relax % \relax, in case this comes first in \halign
  \ifmmode\def\next{\mathpalette\mathbsmash}\else\let\next\makebsmash%
  \fi\next}
\def\makebsmash#1{\setbox0\hbox{#1}\finbsmash}
\def\mathbsmash#1#2{\setbox0\hbox{$\mathsurround0pt#1{#2}$}\finbsmash}
\def\finbsmash{\dp0=0pt \box0}

%------------------------------------------------------------------------------
% Should change the font from 7 to 8 pt (for display style).
\def\restrictDD#1{\Bigr\vert_{#1}}
\def\restrictD#1{\bigr\vert_{#1}}
\def\restrictT#1{\bigr\vert_{#1}}
\def\restrictS#1{\vert_{#1}}
\def\restrictSS#1{\vert_{#1}}
\def\restrict#1{%
 \mathchoice{\restrictD{#1}}{\restrictT{#1}}{\restrictS{#1}}{\restrictSS{#1}}}

\def\restrictDDp#1#2{\Bigr\vert_{#1}^{#2}}
\def\restrictDp#1#2{\bigr\vert_{#1}^{#2}}
\def\restrictTp#1#2{\bigr\vert_{#1}^{#2}}
\def\restrictSp#1#2{\vert_{#1}^{#2}}
\def\restrictSSp#1#2{\vert_{#1}^{#2}}
\def\restrictp#1#2{% Takes subscript, then superscript.
 \mathchoice{\restrictDp{#1}{#2}}{\restrictTp{#1}{#2}}%
 {\restrictSp{#1}{#2}}{\restrictSSp{#1}{#2}}}

% \restrictq is a version of \restrictp which lowers the superscript a bit.
\def\restrictq#1#2{% Takes subscript, then superscript.
 \mathchoice{\Bigr\vert_{#1}^{#2}}{\bigr\vert{}_{#1}^{#2}}%
 {\vert_{#1}^{#2}}{\vert_{#1}^{#2}}}

% A really crummy 2-parameter version of \restrict. Needs fixing.
\def\restrictt#1#2{% Sort of puts one parameter above the other.
 \restrict{\vbox{\baselineskip0pt \parskip0pt %
 \hbox{$\scriptstyle#1$}\hbox{$\scriptstyle#2$}}}}

% - - - - - - - - - - - - - - - - - - - - - - - - - - - - - - - - - - - - - - -
\def\lv{\left\vert}
\def\rv{\right\vert}

\def\delperp{\Delta\kern-1.6pt^\perp}  % \Delta superscript \perp.

%------------------------------------------------------------------------------
% \supsubcent. Centre superscript above subscript.
% #1    Superscript.
% #2    Subscript.
\def\supsubcent[#1#2]{{%
 % Set dimen 0 to the width of #1 in script style.
 \setbox0\hbox{$\scriptstyle#1$}\dimen0\wd0 %
 % Subtract from dimen 0 the width of #2 in script style.
 \setbox0\hbox{$\scriptstyle#2$}\advance\dimen0 by-\wd0 %
 % Set dimen 0 to (width(#1) - width(#2))/2.
 \divide\dimen0 by2 %
 \ifnum\dimen0>0 %
  \dimen1=0pt %
  \dimen2=\dimen0 %
 \else%
  \dimen1=0pt %
  \advance\dimen1 by-\dimen0 %
  \dimen2=0pt %
  \fi%
 {}^{\kern\dimen1{#1}}_{\kern\dimen2{#2}}%
 }}

% - - - - - - - - - - - - - - - - - - - - - - - - - - - - - - - - - - - - - - -
% Christoffel-symbol style macro to balance superscripts and subscripts.
% \def\crfl[#1#2#3]{{\Gamma^{\kern1pt{#1}}_{\kern-1pt{#2}{#3}}}}
% Draw the character #5 with superscript #1 and subscript #2, #3, #4.
\def\crfldg[#1#2#3#4#5]{%
 {% Set dimen 0 to the width of #1 in script style.
 \setbox0\hbox{$\scriptstyle#1$}\dimen0\wd0 %
 % Subtract from dimen 0 the width of #2 #3 in script style.
 \setbox0\hbox{$\scriptstyle{#2}{#3}$}\advance\dimen0-\wd0 %
 % Set dimen 0 to (width(#1) - width(#2) - width(#3))/2.
 \divide\dimen0 by2 %
 % The lr-indent into the lower right of the Gamma.
 \dimen1=-2pt %
 % General shift of everything to the right.
 \dimen4=0.5pt %
 \ifnum\dimen0>\dimen1 %
  % Top kern = 0. Bottom kern = (width(#1) - width(#2) - width(#3))/2.
  \dimen2=0pt \dimen3=\dimen0 %
 \else%
  % Top kern = lr-indent + (width(#2) + width(#3) - width(#1))/2.
  % Bottom kern = lr-indent.
  \dimen2=\dimen1 \advance\dimen2-\dimen0 \dimen3=\dimen1 %
  \fi%
 {{#5}^{\kern\dimen4\kern\dimen2{#1}}%
      _{\kern\dimen4\kern\dimen3{#2}{#3}{#4}}}%
% \hbox{[\the\dimen0,\the\dimen1,\the\dimen2,\the\dimen3]}%
 }}
\def\crfld[#1#2#3#4]{\crfldg[{#1}{#2}{#3}{#4}\Gamma]}
\def\crfl[#1#2#3]{\crfld[{#1}{#2}{#3}\relax]}
\def\crfltilde[#1#2#3]{\crfldg[{#1}{#2}{#3}\relax{\tilde\Gamma}]}
\def\crflchart[#1#2#3#4]{\crfldg[{#1}{#2}{#3}\relax{\Gamma(#4)}]}
\def\crfldchart[#1#2#3#4#5]{\crfldg[{#1}{#2}{#3}{#4}{\Gamma(#5)}]}

%------------------------------------------------------------------------------
% Combination-symbol style-macro to balance superscripts and subscripts.
% Draw the character #3 with superscript #1 and subscript #2.
\def\CombXYZ[#1#2#3]{{%
 % Set dimen 0 to the width of #1 in script style.
 \setbox0\hbox{$\scriptstyle#1$}\dimen0\wd0 %
 % Subtract from dimen 0 the width of #2 in script style.
 \setbox0\hbox{$\scriptstyle#2$}\advance\dimen0-\wd0 %
 % Set dimen 0 to (width(#1) - width(#2))/2.
 \divide\dimen0 by2 %
 % The lr-indent into the lower right of the 'C'.
 \dimen1=-2pt %
 \ifnum\dimen0>0 %
  \dimen2=0pt \dimen3=0pt
 \else
  \ifnum\dimen0>\dimen1 %
   % Top kern = 0. Bottom kern = (width(#1) - width(#2))/2.
   \dimen2=0pt \dimen3=\dimen0 %
  \else%
   % Top kern = lr-indent + (width(#2) - width(#1))/2.
   % Bottom kern = lr-indent.
%   \dimen2=\dimen1 \advance\dimen2-\dimen0 \dimen3=\dimen1 %
   % Top kern = 0.
   % Bottom kern = lr-indent.
   \dimen2=0pt \dimen3=\dimen1 %
   \fi%
  \fi%
 % General shift of everything to the right.
 \dimen4=0pt %
 {#3{}^{\kern\dimen4\kern\dimen2{#1}}%
      _{\kern\dimen4\kern\dimen3{#2}}}%
 }}
\def\CombXY[#1#2]{\CombXYZ[{#1}{#2}C]}
\def\CombX#1#2{\CombXY[{#1}{#2}]}

% Combination/permutation symbols for use in math mode.
% \def\Comb#1#2{C{}^{#1}_{#2}}
\let\Comb\CombX
% Maybe should upgrade the \Perm macro like the \Comb macro?
\def\Perm#1#2{P{}^{#1}_{#2}}

%------------------------------------------------------------------------------
\def\pdv#1#2{\frac{\partial#1}{\partial#2}}
\def\pdl#1#2{\partial#1/\partial#2}
\def\pd#1#2{%
 \mathchoice{\pdv{#1}{#2}}{\pdl{#1}{#2}}{\pdv{#1}{#2}}{\pdv{#1}{#2}}}
\def\ebold{\bold{e}}
\def\smallhalf{{\textstyle\frac12}}
\def\longto{\mathop{\longrightarrow}\limits}
\def\longfrom{\mathop{\longleftarrow}\limits}
\def\rlmap{\mathop{\normalbaselines\vcenter{\hbox{$\longrightarrow$}%
 \kern-7pt\hbox{$\longleftarrow$}}}\limits}

% Colon to separate a function name from a following domain symbol.
\def\from{\mathop{:}}

\def\komb#1{\left\{\vcenter{\halign{##\hfil\cr#1\crcr}}\right.}
\def\Komb#1{$\komb{#1}$}
\def\Prob{\mathop{\rm Prob}}

% \def\conv{\mathop{\rm conv}}
% \def\conv{\mathop{\eusm C}\nolimits}
\def\conv{\mathord{\eusm C}}
\def\codim{\textop{codim}}
\def\sign{\textop{sign}}
\def\diam{\textop{diam}}
\def\floor{\textop{floor}}
\def\fracpart{\textop{frac}}
\def\ceiling{\textop{ceiling}}
\def\round{\textop{round}}
\def\length{\textop{length}}
\def\cosec{\textop{cosec}}

\def\opgcd{\textop{gcd}}
\def\opparity{\textop{parity}}
\def\opperm{\textop{perm}}

% Functions.
\def\Image{\textop{Img}}
\def\Dom{\textop{Dom}}
\let\Domain\Dom
\def\Range{\textop{Range}}

% Support of a function.
% The closure of the subset of the domain where the function is non-zero.
\def\supp{\textop{supp}}

% - - - - - - - - - - - - - - - - - - - - - - - - - - - - - - - - - - - - - - -
% Lists.
\def\concat{\textop{concat}}

% \def\opinsert{\mathop{\rm insert}\nolimits}
\def\opinsert{\textopdl{insert}}
% \def\opomit{\mathop{\rm omit}\nolimits}
\def\opomit{\textopdl{omit}}
\def\opsubs{\textopdl{subs}}
\def\opsubseq{\textopdl{subseq}}
\def\opswap{\textopdl{swap}}

% - - - - - - - - - - - - - - - - - - - - - - - - - - - - - - - - - - - - - - -
% Set-morphisms and cardinality.
\def\Inj{\textop{Inj}}
\def\Sur{\textop{Surj}}
\def\Bij{\textop{Bij}}
\def\Enum{\textop{Enum}}

% Algebra.
\def\Tr{\textop{Tr}}
\def\ad{\textop{ad}}    % Adjoint.

\def\Hom{\textop{Hom}}  % Space of homomorphisms. (X -> Y)
\def\End{\textop{End}}  % Space of endomorphisms. (X -> X)
\def\Iso{\textop{Iso}}  % Space of isomorphisms. (bijective X -> Y)
\def\Aut{\textop{Aut}}  % Space of automorphisms. (bijective X -> X)
\def\Mon{\textop{Mon}}  % Space of monomorphisms. (injective X -> Y)
\def\Epi{\textop{Epi}}  % Space of epimorphisms. (surjective X -> Y)

% Linear spaces.
\def\Lin{\textop{Lin}}
% \let\lin\Lin
\def\Span{\textop{span}}

% Topology.
\def\Top{\textop{Top}}
\def\Nbhd{\textop{Nbhd}}
\def\Int{\textop{Int}}
\def\Clos{\textop{Clos}}
\def\Ext{\textop{Ext}}
\def\Bdy{\textop{Bdy}}
% Topological separation classes.
\def\tc{{\bf T}}

% Manifolds.
\def\atlas{\textop{atlas}}
\def\atlasbar{\textop{\smash{\overline{\vphantom{t}\smash{\rm atlas}}}}}
% \def\atlasbar{\textop{\smash{\overline{\rm atlas}}}}
% \def\atlasbar{\textop{\overline{\rm atlas}}}
% \def\atlasbarx{\smash{\textop{\overline{\rm atlas}}}}

\def\id{\textord{id}}       % Ordinary to fix spacing for following \circ.
\def\pound{{\it\$}}
\def\mlin{{\rsfs L}}        % Set of multilinear maps. Needs font rsfs10.

% - - - - - - - - - - - - - - - - - - - - - - - - - - - - - - - - - - - - - - -
% Characters with skewed (shifted) accents.
\def\hatphi{\skew5\hat\phi}
\def\checkphi{\skew5\check\phi}
\def\tildephi{\skew5\tilde\phi}

% Negative skew for accents. Adapted from TeXbook, page 359.
\def\skewNEG#1#2#3{{#2{#3\mkern-#1mu}\mkern#1mu}{}}

% This is what is in the TeXbook.
% \def\skew#1#2#3{{#2{#3\mkern#1mu}\mkern-#1mu}{}}

% This is what I found in plain.tex.
% \def\skew#1#2#3{{\muskip\z@#1mu\divide\muskip\z@\tw@ \mkern\muskip\z@
%     #2{\mkern-\muskip\z@{#3}\mkern\muskip\z@}\mkern-\muskip\z@}{}}

% - - - - - - - - - - - - - - - - - - - - - - - - - - - - - - - - - - - - - - -
% Miscellaneous things.
\def\Cpp/{{\rm C\raise.2ex\hbox{++}}}   % C++ programming language.

% North-eastern square for use as a NOT-operator (maybe).
\def\nboxitNE#1#2{% This is a crummy macro because of unknown TeX box padding.
 \vbox{\hrule height#2 \kern-#2\hbox{\vbox{#1}\vrule width#2}}}
\def\squareNE{\nboxitNE{\hbox{\vbox{\kern6pt}\kern6pt}}{0.25pt}}
\def\squareNEsmall{\nboxitNE{\hbox{\vbox{\kern4pt}\kern4pt}}{0.2pt}}
% Substitute for logical negation operator symbol.
\let\Lnot\squareNE
\let\Lnotsmall\squareNEsmall

% Superimposed \land and \lor to get an exclusive OR symbol candidate.
\def\lorand{\mathbin{\lor{\setbox0\hbox{$\land$}\hskip-\wd0}\land}}%
\def\tslorand{\ts\lorand\ts}

% A V-bar macro for exclusive OR.
\def\lbarvee{\mathbin{\bar\vee}}
\def\tslbarvee{\ts\lbarvee\ts}

% Medium size wedge and vee. Very kludgy!
\def\opmedwedge{\mathop{\hbox{\eusm\char"5E}}\limits}
\def\opmedvee{\mathop{\hbox{\eusm\char"5F}}\limits}
% An atom-version with zero space on the right.
\def\medwedge{\hbox{\eusm\char"5E}}
\def\medvee{\hbox{\eusm\char"5F}}

% 2009-2-4: Moved to akfont.tex.
% \let\lxor\lbarvee
% \let\tslxor\tslbarvee

% Symmetric set difference. (Corresponds to logical exclusive OR.)
\def\setdiff{\mathbin{\bigtriangleup}}
\def\tssetdiff{\ts\mathbin{\bigtriangleup}\ts}

% Some logical operators.
\def\sstroke{\mathbin{|}}           % Sheffer stroke.
\def\arrownand{\mathbin{\uparrow}}  % Alternative symbol for Sheffer stroke.
\def\Qarrow{\mathbin{\downarrow}}   % Quine arrow.

% \let\lnand\sstroke
\let\lnand\arrownand
\let\lnor\Qarrow

\def\tslnand{\ts\lnand\ts}
\def\tslnor{\ts\lnor\ts}
