% tex/conc/mp/mpdemo.tex   2018-2-2   Alan U. Kennington.
% $Id: tex/conc/mp/mpdemo.tex 631e2d0fd4 2017-09-23 03:47:28Z Alan U. Kennington $
% Copyright (C) 2016, Alan U. Kennington.
% Demo document for differential geometry diagrams created using MetaPost.

% See this link for info on displaying MetaPost with Ghostcript:
% http://www.tex.ac.uk/cgi-bin/texfaq2html?keyword=&question=76

\input akmath
\input dgspell
\input epsf

%==============================================================================
\parindent0pt
% Adjust the printing area for A4 paper.
\hoffset=-2.838mm
\vsize=29.73017788cm
\advance\vsize by-2.54cm            % Top margin
\advance\vsize by-2.54cm            % Bottom margin
\advance\vsize by-0.9cm             % Allowance for page number in footer.

% Calculate the number of hours and number of minutes (24 hour format):
\newcount\inthour
\newcount\inthoursixty
\newcount\intminute
\inthour=\time
\divide\inthour60
\intminute=\time
\inthoursixty=\inthour
\multiply\inthoursixty60
\advance\intminute-\inthoursixty

% Text representations of hours and minutes (24 hour format):
\def\zmonth{\ifnum\month<10\relax0\fi\number\month}
\def\zday{\ifnum\day<10\relax0\fi\number\day}
\def\zhour{\ifnum\inthour<10\relax0\fi\number\inthour}
\def\zminute{\ifnum\intminute<10\relax0\fi\number\intminute}

% Plain TeX hyperlinks.
{\catcode`\^=6 \catcode`\#=12 \gdef\PreHatch^1{#^1}}
\def\LinkName#1{\special{html:<a name="#1">}\special{html:</a>}}
\def\LinkHref#1#2{\special{html:<a href="\PreHatch{#1}">}#2\special{html:</a>}}
\def\LinkHrefExt#1#2{\special{html:<a href="#1">}#2\special{html:</a>}}

%==============================================================================
This is Alan U.\ Kennington's little gallery of MetaPost diagrams, created for
\LinkHrefExt{http://www.geometry.org/tex/conc/dg.html}{my differential geometry
book}. These diagrams demonstrate the capabilities of MetaPost, a graphics tool
which combines mathematical specification of graphical elements (based on Donald
E.\ Knuth's MetaFont) with text specified in plain \TeX\ (Donald E.\ Knuth's
state-of-the-art typesetting system). This is exactly what every mathematician
needs for illustrations. Many thanks to John D.\ Hobby, the creator of MetaPost.

Source files for all diagrams in this document are available at:
\LinkHrefExt{http://www.geometry.org/tex/conc/mp/}{{\tt
http://www.geometry.org/tex/conc/mp/}}
\hfill\break
Gaps in the diagram numbering scheme are due to filename changes,
not due to omissions.
\hfill\break
Typesetting date/time (local) of this document:
{\tt \number\year-\zmonth-\zday\space\zhour:\zminute}.

% Local macros.
\def\secteject{\vfill\eject}
\def\filleject{\vfill\eject}
% Temp:
% \def\filleject{}
% \raggedbottom is not very good. It turns 78 pages into 80 pages somehow.
% \raggedbottom

\vskip2mm
%==============================================================================
\edef\SECTarrow{\the\pageno}\LinkName{arrow}

Diagram arrow1. Tensor product and multilinear maps.
$$
\centerline{\epsfbox{arrow1.1}}
$$

Diagram arrow2. Uniqueness of tensor product.
$$
\centerline{\epsfbox{arrow2.1}}
$$

Diagram arrow3. Map between tensor product spaces.
$$
\centerline{\epsfbox{arrow3.1}}
$$

Diagram arrow4. Transformation of tangent vectors of curves under
diffeomorphisms.
$$
\centerline{\epsfbox{arrow4.1}}
$$

Diagram arrow5. Transformation of differentials of functions under
diffeomorphisms.
$$
\centerline{\epsfbox{arrow5.1}}
$$

% - - - - - - - - - - - - - - - - - - - - - - - - - - - - - - - - - - - - - - -
\filleject

Diagram arrow6. Tensor product metadefinition and multilinear maps.
$$
\centerline{\epsfbox{arrow6.1}}
$$

Diagram arrow7. Relations between tensor spaces.
$$
\centerline{\epsfbox{arrow7.1}}
$$

Diagram arrow8. Relations between antisymmetric tensor spaces.
$$
\centerline{\epsfbox{arrow8.1}}
$$

Diagram arrow9. Relations between tensor spaces for linear space families.
$$
\centerline{\epsfbox{arrow9.1}}
$$

Diagram arrow10. Row and column vector maps.
$$
\centerline{\epsfbox{arrow10.1}}
$$

Diagram arrow11. Universal \factoris/ation map for tensor spaces.
$$
\centerline{\epsfbox{arrow11.1}}
$$

% - - - - - - - - - - - - - - - - - - - - - - - - - - - - - - - - - - - - - - -
\filleject

Diagram arrow12. Maps for curves and real functions on a differentiable
manifold.
$$
\centerline{\epsfbox{arrow12.1}}
$$

Diagram arrow13. Universal \factoris/ation dual map for multilinear functions.
$$
\centerline{\epsfbox{arrow13.1}}
$$

Diagram arrow14. Application of universal \factoris/ation dual map.
$$
\centerline{\epsfbox{arrow14.1}}
$$

Diagram arrow15. Isomorphisms and canonical multilinear maps for tensor spaces
for linear space families.
$$
\centerline{\epsfbox{arrow15.1}}
$$

Diagram arrow16. Application of universal \factoris/ation dual map transpose.
$$
\centerline{\epsfbox{arrow16.1}}
$$

Diagram arrow17. Application of universal \factoris/ation dual map for
multilinear functions.
$$
\centerline{\epsfbox{arrow17.1}}
$$

% - - - - - - - - - - - - - - - - - - - - - - - - - - - - - - - - - - - - - - -
\filleject

Diagram arrow18. Canonical multilinear map domains and ranges.
$$
\centerline{\epsfbox{arrow18.1}}
$$

Diagram arrow19. Canonical multilinear map domains and ranges. Plus
isomorphisms.
$$
\centerline{\epsfbox{arrow19.1}}
$$

Diagram arrow20. Tensor space duals, including canonical multilinear maps.
$$
\centerline{\epsfbox{arrow20.1}}
$$

Diagram arrow21. Norm on a module over a ring.
$$
\centerline{\epsfbox{arrow21.1}}
$$

Diagram arrow22. Spaces and maps for function order definition for module over a
ring.
$$
\centerline{\epsfbox{arrow22.1}}
$$

Diagram arrow23. Spaces and maps for function order definition for linear
spaces.
$$
\centerline{\epsfbox{arrow23.1}}
$$

% - - - - - - - - - - - - - - - - - - - - - - - - - - - - - - - - - - - - - - -
\filleject

Diagram arrow24. Norm on a linear space over a field.
$$
\centerline{\epsfbox{arrow24.1}}
$$

Diagram arrow25. Sets and operations for modules.
$$
\centerline{\epsfbox{arrow25.1}}
$$

Diagram arrow26. Sets and operations for affine spaces over modules.
$$
\centerline{\epsfbox{arrow26.1}}
$$

Diagram arrow27. Sets and operations for affine spaces over modules, showing
lines.
$$
\centerline{\epsfbox{arrow27.1}}
$$

% - - - - - - - - - - - - - - - - - - - - - - - - - - - - - - - - - - - - - - -
\filleject

Diagram arrow28. Maps for curves and real functions on a differentiable
manifold. Based on arrow12.
$$
\centerline{\epsfbox{arrow28.1}}
$$

Diagram arrow29. Sets and operations for single-set algebraic structures and
modules.
$$
\centerline{\epsfbox{arrow29.1}}
$$

Diagram arrow30. Mixed tensor product space definition.
$$
\centerline{\epsfbox{arrow30.1}}
$$

Diagram arrow31. Mixed tensor product spaces isomorphism.
$$
\centerline{\epsfbox{arrow31.1}}
$$

% - - - - - - - - - - - - - - - - - - - - - - - - - - - - - - - - - - - - - - -
\filleject

Diagram arrow32. Notations for spaces of multilinear maps.
$$
\centerline{\epsfbox{arrow32.1}}
$$

Diagram arrow33. Notations for spaces of multilinear functions.
$$
\centerline{\epsfbox{arrow33.1}}
$$

Diagram arrow34. Notations for tensor spaces.
$$
\centerline{\epsfbox{arrow34.1}}
$$

Diagram arrow35. Abbreviated diagrams for notations for tensor spaces.
$$
\centerline{\epsfbox{arrow35.1}}
$$

Diagram arrow36. Tensor space duals and canonical multilinear maps.
$$
\centerline{\epsfbox{arrow36.1}}
$$

% - - - - - - - - - - - - - - - - - - - - - - - - - - - - - - - - - - - - - - -
\filleject

Diagram arrow37. Tensor space duals and canonical multilinear maps. Showing
isomorphisms.
$$
\centerline{\epsfbox{arrow37.1}}
$$

Diagram arrow38. Tensor spaces based on a single linear space.
$$
\centerline{\epsfbox{arrow38.1}}
$$

Diagram arrow39. Riemann curvature tensor spaces/maps for an affine connection.
$$
\centerline{\epsfbox{arrow39.1}}
$$

Diagram arrow40. Riemann curvature tensor spaces/maps for an affine connection.
Two representations.
$$
\centerline{\epsfbox{arrow40.1}}
$$

% - - - - - - - - - - - - - - - - - - - - - - - - - - - - - - - - - - - - - - -
\filleject

Diagram arrow41. Riemann curvature tensor spaces/maps for a Riemannian manifold.
$$
\centerline{\epsfbox{arrow41.1}}
$$

Diagram arrow42. Entity-relationship model for elementary category theory.
$$
\centerline{\epsfbox{arrow42.1}}
$$

Diagram arrow52. Isomorphism between left actions on points and right actions on
the group.
$$
\centerline{\epsfbox{arrow52.1}}
$$

Diagram arrow53. Dualities and isomorphisms between antisymmetric tensor spaces.
$$
\centerline{\epsfbox{arrow53.1}}
$$

Diagram arrow54. Dualities between antisymmetric tensor spaces.
$$
\centerline{\epsfbox{arrow54.1}}
$$

Diagram arrow55. Dualities and isomorphisms between antisymmetric tensor spaces.
$$
\centerline{\epsfbox{arrow55.1}}
$$

% - - - - - - - - - - - - - - - - - - - - - - - - - - - - - - - - - - - - - - -
% \filleject

%==============================================================================
\secteject
\edef\SECTaxiom{\the\pageno}\LinkName{axiom}

Diagram axiom1. ZF set theory axiom of regularity.
$$
\centerline{\epsfbox{axiom1.1}}
$$

Diagram axiom2. Axiom of choice. Choice of function values from a relation.
$$
\centerline{\epsfbox{axiom2.1}}
$$

Diagram axiom3. Axiom of choice. Function choosing elements of subsets of a set.
$$
\centerline{\epsfbox{axiom3.1}}
$$

Diagram axiom4. Axiom of choice. Choice of elements of sets of a disjoint
collection.
$$
\centerline{\epsfbox{axiom4.1}}
$$

Diagram axiom5. ZF axiom of replacement.
$$
\centerline{\epsfbox{axiom5.1}}
$$

Diagram axiom6. ZF axiom of extension.
$$
\centerline{\epsfbox{axiom6.1}}
$$

% - - - - - - - - - - - - - - - - - - - - - - - - - - - - - - - - - - - - - - -
\filleject

Diagram axiom7. ZF axiom of union. Two sets only.
$$
\centerline{\epsfbox{axiom7.1}}
$$

Diagram axiom8. ZF axiom of union. General case.
$$
\centerline{\epsfbox{axiom8.1}}
$$

Diagram axiom9. Impossible axiom-of-choice set $C$ if $A_1\bcap A_2=\emptyset$
and~$A_1\bcup A_2\subseteq A_3$. (Based on axiom3.)
$$
\centerline{\epsfbox{axiom9.1}}
$$

Diagram axiom10. ZF axioms of regularity and infinity bounding the finite
ordinal numbers.
$$
\centerline{\epsfbox{axiom10.1}}
$$

Diagram axiom11. The ZF regularity axiom versus Russell's paradox.
$$
\centerline{\epsfbox{axiom11.1}}
$$

%==============================================================================
\secteject
\edef\SECTbox{\the\pageno}\LinkName{box}

Diagram box1. Medieval trivium and quadrivium.
$$
\centerline{\epsfbox{box1.1}}
$$

Diagram box2. Magma, bedrock and sedimentary layers.
$$
\centerline{\epsfbox{box2.1}}
$$

Diagram box3. Medieval trivium and quadrivium. Inverted order to box1.
$$
\centerline{\epsfbox{box3.1}}
$$

% - - - - - - - - - - - - - - - - - - - - - - - - - - - - - - - - - - - - - - -
\filleject

Diagram box4. Finite naive mathematics and infinite rigorous mathematics.
$$
\centerline{\epsfbox{box4.1}}
$$

Diagram box5. Universe set interpreted in context of computer
directories/folders.
$$
\centerline{\epsfbox{box5.1}}
$$

Diagram box6. The Russell's paradox set interpreted as computer
directories/folders.
$$
\centerline{\epsfbox{box6.1}}
$$

%==============================================================================
\secteject
\edef\SECTcalc{\the\pageno}\LinkName{calc}

Diagram calc1. Inverse function theorem. Proof that the inverse is
differentiable.
$$
\centerline{\epsfbox{calc1.1}}
$$

Diagram calc2. Chain rule for partial derivatives.
$$
\centerline{\epsfbox{calc2.1}}
$$

%==============================================================================
\secteject
\edef\SECTchart{\the\pageno}\LinkName{chart}

Diagram chart1. Chart for a topological manifold.
$$
\centerline{\epsfbox{chart1.1}}
$$

Diagram chart2. Transition map for two coordinate maps for a differentiable
manifold.
$$
\centerline{\epsfbox{chart2.1}}
$$

Diagram chart3. Definition of differentiability of $\reals^m$-valued functions
on a manifold.
$$
\centerline{\epsfbox{chart3.1}}
$$

% - - - - - - - - - - - - - - - - - - - - - - - - - - - - - - - - - - - - - - -
\filleject

Diagram chart4. Definition of differentiability of $\reals^m$-valued functions
on a manifold.
$$
\centerline{\epsfbox{chart4.1}}
$$

Diagram chart5. Definition of tangent vectors on manifolds.
$$
\centerline{\epsfbox{chart5.1}}
$$

Diagram chart6. Charts for a topological manifold.
$$
\centerline{\epsfbox{chart6.1}}
$$

% - - - - - - - - - - - - - - - - - - - - - - - - - - - - - - - - - - - - - - -
\filleject

Diagram chart7. Atlas for a topological manifold.
$$
\centerline{\epsfbox{chart7.1}}
$$

Diagram chart8. Map for the collage of two charts of a manifold.
$$
\centerline{\epsfbox{chart8.1}}
$$

Diagram chart9. Map for the collage of two charts of a manifold, with collage
set.
$$
\centerline{\epsfbox{chart9.1}}
$$

% - - - - - - - - - - - - - - - - - - - - - - - - - - - - - - - - - - - - - - -
\filleject

Diagram chart10. Modified version of chart4. Differentiability of
$\reals^m$-valued function on a manifold.
$$
\centerline{\epsfbox{chart10.1}}
$$

Diagram chart11. Modified version of chart10. Differentiability of
$\reals$-valued function on a manifold.
$$
\centerline{\epsfbox{chart11.1}}
$$

Diagram chart12. Chart-dependence of rectifiability in topological manifolds.
$$
\centerline{\epsfbox{chart12.1}}
$$

Diagram chart13. Maps for tangent vector and covector.
$$
\centerline{\epsfbox{chart13.1}}
$$

% - - - - - - - - - - - - - - - - - - - - - - - - - - - - - - - - - - - - - - -
\filleject

Diagram chart14. Maps for a curve in a differentiable manifold.
$$
\centerline{\epsfbox{chart14.1}}
$$

Diagram chart15. Comparison of Cartesian charts for Euclidean space and
differentiable manifolds.
$$
\centerline{\epsfbox{chart15.1}}
$$

Diagram chart16. Chart transition rules for bidirectional and unidirectional
tangent line vectors.
$$
\centerline{\epsfbox{chart16.1}}
$$

% - - - - - - - - - - - - - - - - - - - - - - - - - - - - - - - - - - - - - - -
\filleject

Diagram chart17. Chart transition rules for tangent line vectors on $C^1$
differentiable manifolds.
$$
\centerline{\epsfbox{chart17.1}}
$$

Diagram chart18. Tangent vector field of a differentiable curve.
$$
\centerline{\epsfbox{chart18.1}}
$$

%==============================================================================
\secteject
\edef\SECTclass{\the\pageno}\LinkName{class}

Diagram class1. Figures 88 and 89 from Euler's ``Introductio in analysin
infinitorum'', 1748.
$$
\centerline{\epsfbox{class1.1}}
$$

Diagram class2. Classical definitions of sine, cosine and tangent.
$$
\centerline{\epsfbox{class2.1}}
$$

Diagram class3. Tangent to Archimedes spiral.
$$
\centerline{\epsfbox{class3.1}}
$$

Diagram class4. Chords approximating a tangent in the limit.
$$
\centerline{\epsfbox{class4.1}}
$$

% - - - - - - - - - - - - - - - - - - - - - - - - - - - - - - - - - - - - - - -
\filleject

Diagram class5. Comparison of differential quotient and absolute value
approaches.
$$
\centerline{\epsfbox{class5.1}}
$$

Diagram class6. Construction of Cartesian coordinates in Euclidean geometry.
$$
\centerline{\epsfbox{class6.1}}
$$

Diagram class7. Interpretation of a parallelogram law.
$$
\centerline{\epsfbox{class7.1}}
$$

Diagram class8. Division of a line segment in a given ratio using rule and
compass.
$$
\centerline{\epsfbox{class8.1}}
$$

% - - - - - - - - - - - - - - - - - - - - - - - - - - - - - - - - - - - - - - -
\filleject

Diagram class9. Construction of contravariant and covariant coordinates in
Euclidean geometry.
$$
\centerline{\epsfbox{class9.1}}
$$

%==============================================================================
\secteject
\edef\SECTconnmap{\the\pageno}\LinkName{connmap}

Diagram connmap1. Connection maps.
$$
\centerline{\epsfbox{connmap1.1}}
$$

Diagram connmap2. Lift of a vector field by a connection.
$$
\centerline{\epsfbox{connmap2.1}}
$$

Diagram connmap3. Connection maps, including direction $v$.
$$
\centerline{\epsfbox{connmap3.1}}
$$

% - - - - - - - - - - - - - - - - - - - - - - - - - - - - - - - - - - - - - - -
\filleject

Diagram connmap4. Subtraction of connection from second derivative to give a
vertical vector.
$$
\centerline{\epsfbox{connmap4.1}}
$$

Diagram connmap5. Subtraction of connection from derivative of vector field to
give a vertical vector.
$$
\centerline{\epsfbox{connmap5.1}}
$$

Diagram connmap6. Connection on an ordinary \fibre/ bundle.
$$
\centerline{\epsfbox{connmap6.1}}
$$

Diagram connmap7. Maps for connection on ordinary \fibre/ bundle.
$$
\centerline{\epsfbox{connmap7.1}}
$$

% - - - - - - - - - - - - - - - - - - - - - - - - - - - - - - - - - - - - - - -
\filleject

Diagram connmap8. Copying a parallelism between associated \fibre/ bundles.
$$
\centerline{\epsfbox{connmap8.1}}
$$

Diagram connmap9. Flow of vector $V$ for a vector field $X$.
$$
\centerline{\epsfbox{connmap9.1}}
$$

Diagram connmap10. Lie derivative of vector field $Y$ with respect to vector
field~$X$.
$$
\centerline{\epsfbox{connmap10.1}}
$$

Diagram connmap11. Right action of structure group on a principal \fibre/
bundle.
$$
\centerline{\epsfbox{connmap11.1}}
$$

% - - - - - - - - - - - - - - - - - - - - - - - - - - - - - - - - - - - - - - -
\filleject

Diagram connmap12. Torsion definition difficulty on an abstract ordinary \fibre/
bundle. Based on connmap6.
$$
\centerline{\epsfbox{connmap12.1}}
$$

Diagram connmap13. Torsion definition difficulty on an abstract ordinary \fibre/
bundle.
$$
\centerline{\epsfbox{connmap13.1}}
$$

Diagram connmap14. Affine connection on a tangent bundle.
$$
\centerline{\epsfbox{connmap14.1}}
$$

% - - - - - - - - - - - - - - - - - - - - - - - - - - - - - - - - - - - - - - -
\filleject

Diagram connmap15. Affine connection on a vector bundle.
$$
\centerline{\epsfbox{connmap15.1}}
$$

Diagram connmap16. Infinitesimal action of general linear Lie transformation
group.
$$
\centerline{\epsfbox{connmap16.1}}
$$

%==============================================================================
\secteject
\edef\SECTfibdiag{\the\pageno}\LinkName{fibdiag}

Diagram fibdiag1. Maps for non-topological fibrations, and ordinary and
principal \fibre/ bundles.
$$
\centerline{\epsfbox{fibdiag1.1}}
$$

Diagram fibdiag2. Maps and spaces for tangent bundles from the \fibre/ bundle
perspective.
$$
\centerline{\epsfbox{fibdiag2.1}}
$$

Diagram fibdiag3. Cyclic construction of tangent bundles and structure groups
on manifolds.
$$
\centerline{\epsfbox{fibdiag3.1}}
$$

Diagram fibdiag4. Spaces and maps for baseless figure bundles.
$$
\centerline{\epsfbox{fibdiag4.1}}
$$

% - - - - - - - - - - - - - - - - - - - - - - - - - - - - - - - - - - - - - - -
\filleject

Diagram fibdiag5. Spaces and maps for baseless figure bundles. Multiple
ordinary bundles.
$$
\centerline{\epsfbox{fibdiag5.1}}
$$

Diagram fibdiag6. Spaces and maps for induced map of a differentiable \fibre/
bundle.
$$
\centerline{\epsfbox{fibdiag6.1}}
$$

Diagram fibdiag7. Spaces and maps for combined topological figure/frame bundles.
$$
\centerline{\epsfbox{fibdiag7.1}}
$$

Diagram fibdiag8. Combined topological figure/frame bundles for multiple
ordinary bundles.
$$
\centerline{\epsfbox{fibdiag8.1}}
$$

% - - - - - - - - - - - - - - - - - - - - - - - - - - - - - - - - - - - - - - -
\filleject

Diagram fibdiag9. Maps/spaces for topological \fibre/ bundle
group/observer/observation/state.
$$
\centerline{\epsfbox{fibdiag9.1}}
$$

Diagram fibdiag10. Primal and dual vector bundle association.
$$
\centerline{\epsfbox{fibdiag10.1}}
$$

Diagram fibdiag11. Maps and spaces for linearity of connections on vector
bundles.
$$
\centerline{\epsfbox{fibdiag11.1}}
$$

%==============================================================================
\secteject
\edef\SECTfibmap{\the\pageno}\LinkName{fibmap}

Diagram fibmap1. Vector bundle maps and spaces.
$$
\centerline{\epsfbox{fibmap1.1}}
$$

Diagram fibmap2. Ordinary \fibre/ bundle maps and spaces.
$$
\centerline{\epsfbox{fibmap2.1}}
$$

Diagram fibmap3. Principal \fibre/ bundle maps and spaces. Adapted from fibmap9.
$$
\centerline{\epsfbox{fibmap3.1}}
$$

Diagram fibmap7. \Fibre/ bundle maps, including product of projection and
coordinate map.
$$
\centerline{\epsfbox{fibmap7.1}}
$$

% - - - - - - - - - - - - - - - - - - - - - - - - - - - - - - - - - - - - - - -
\filleject

Diagram fibmap8. Principal \fibre/ bundle maps and spaces.
$$
\centerline{\epsfbox{fibmap8.1}}
$$

Diagram fibmap9. Principal \fibre/ bundle maps and spaces.
$$
\centerline{\epsfbox{fibmap9.1}}
$$

Diagram fibmap10. Group invariance of \fibre/ chart transition maps.
$$
\centerline{\epsfbox{fibmap10.1}}
$$

Diagram fibmap11. \Fibre/ bundle map -- between two \fibre/ bundles.
$$
\centerline{\epsfbox{fibmap11.1}}
$$

% - - - - - - - - - - - - - - - - - - - - - - - - - - - - - - - - - - - - - - -
\filleject

Diagram fibmap12. Maps for groupless \fibre/ bundle.
$$
\centerline{\epsfbox{fibmap12.1}}
$$

Diagram fibmap13. Maps for groupless \fibre/ bundle with \fibre/ $F$.
$$
\centerline{\epsfbox{fibmap13.1}}
$$

Diagram fibmap14. Transition maps for groupless \fibre/ bundle with \fibre/ $F$.
$$
\centerline{\epsfbox{fibmap14.1}}
$$

% - - - - - - - - - - - - - - - - - - - - - - - - - - - - - - - - - - - - - - -
\filleject

Diagram fibmap15. Transition maps for groupless \fibre/ bundle with \fibre/ $F$.
Mirror image of fibmap14.
$$
\centerline{\epsfbox{fibmap15.1}}
$$

Diagram fibmap16. Principal \fibre/ bundle maps and spaces. Adapted from
fibmap3.
$$
\centerline{\epsfbox{fibmap16.1}}
$$

Diagram fibmap17. \Fibre/ bundle maps and spaces.
$$
\centerline{\epsfbox{fibmap17.1}}
$$

% - - - - - - - - - - - - - - - - - - - - - - - - - - - - - - - - - - - - - - -
\filleject

Diagram fibmap18. Projection maps and charts for \fibre/ bundles.
$$
\centerline{\epsfbox{fibmap18.1}}
$$

Diagram fibmap19. Maps/spaces for \fibre/ bundle homomorphism. Based on
connmap7.
$$
\centerline{\epsfbox{fibmap19.1}}
$$

Diagram fibmap20. Maps/spaces for \fibre/ bundle association. Based on fibmap19.
$$
\centerline{\epsfbox{fibmap20.1}}
$$

% - - - - - - - - - - - - - - - - - - - - - - - - - - - - - - - - - - - - - - -
\filleject

Diagram fibmap21. Maps/spaces for $C^0$ associated \fibre/ bundles. Based on
fibmap20.
$$
\centerline{\epsfbox{fibmap21.1}}
$$

Diagram fibmap22. Associated \fibre/ bundle construction, orbit space method.
Based on fibmap21.
$$
\centerline{\epsfbox{fibmap22.1}}
$$

Diagram fibmap24. Chart styles for the definition of a groupless \fibre/ bundle
with \fibre/~$F$.
$$
\centerline{\epsfbox{fibmap24.1}}
$$

% - - - - - - - - - - - - - - - - - - - - - - - - - - - - - - - - - - - - - - -
\filleject

Diagram fibmap25. Combined \fibre//frame bundle maps and spaces.
$$
\centerline{\epsfbox{fibmap25.1}}
$$

Diagram fibmap26. Maps for differentiable groupless \fibre/ bundle. Based on
fibmap12.
$$
\centerline{\epsfbox{fibmap26.1}}
$$

Diagram fibmap27. Maps for differentiable groupless \fibre/ bundle with
\fibre/~$F$. Based on fibmap13.
$$
\centerline{\epsfbox{fibmap27.1}}
$$

% - - - - - - - - - - - - - - - - - - - - - - - - - - - - - - - - - - - - - - -
\filleject

Diagram fibmap28. Differentiable \fibre/ bundle maps, including product of
projection and coordinate map.
$$
\centerline{\epsfbox{fibmap28.1}}
$$

Diagram fibmap29. Differentiable principal \fibre/ bundle maps and spaces. Based
on fibmap16.
$$
\centerline{\epsfbox{fibmap29.1}}
$$

% - - - - - - - - - - - - - - - - - - - - - - - - - - - - - - - - - - - - - - -
\filleject

Diagram fibmap30. Differentiable \fibre/ bundle maps for a tangent bundle.
$$
\centerline{\epsfbox{fibmap30.1}}
$$

Diagram fibmap31. Correspondence between connection on fibration and vector
field on fibre space.
$$
\centerline{\epsfbox{fibmap31.1}}
$$

Diagram fibmap32. Principal \fibre/ bundle maps and spaces.
$$
\centerline{\epsfbox{fibmap32.1}}
$$

% - - - - - - - - - - - - - - - - - - - - - - - - - - - - - - - - - - - - - - -
\filleject

Diagram fibmap33. Maps and spaces for associated topological fibre bundle
collage construction.
$$
\centerline{\epsfbox{fibmap33.1}}
$$

Diagram fibmap34. Maps and spaces for associated topological fibre bundle
orbit-space construction.
$$
\centerline{\epsfbox{fibmap34.1}}
$$

Diagram fibmap35. Maps for differentiable fibration with \fibre/ space~$F$.
(Based on fibmap27.)
$$
\centerline{\epsfbox{fibmap35.1}}
$$

% - - - - - - - - - - - - - - - - - - - - - - - - - - - - - - - - - - - - - - -
\filleject

Diagram fibmap36. Differentiable fibration with implicit \fibre/ space. (Based
on fibmap26.)
$$
\centerline{\epsfbox{fibmap36.1}}
$$

Diagram fibmap37. Horizontal lift function, horizontal component map, and
horizontal subspace.
$$
\centerline{\epsfbox{fibmap37.1}}
$$

Diagram fibmap38. Horizontal lift/component/subspace, connection form and
infinitesimal group action.
$$
\centerline{\epsfbox{fibmap38.1}}
$$

Diagram fibmap39. Horizontal and vertical components of vectors on \fibre/
bundles.
$$
\centerline{\epsfbox{fibmap39.1}}
$$

% - - - - - - - - - - - - - - - - - - - - - - - - - - - - - - - - - - - - - - -
\filleject

Diagram fibmap40. Horizontal and vertical components of vectors on principal
\fibre/ bundles.
$$
\centerline{\epsfbox{fibmap40.1}}
$$

Diagram fibmap41. Left action map by a principal bundle element.
$$
\centerline{\epsfbox{fibmap41.1}}
$$

Diagram fibmap42. Differential of left action map by a principal bundle element.
$$
\centerline{\epsfbox{fibmap42.1}}
$$

% - - - - - - - - - - - - - - - - - - - - - - - - - - - - - - - - - - - - - - -
\filleject

Diagram fibmap43. Maps and spaces for associated topological tangent vector and
covector bundles.
$$
\centerline{\epsfbox{fibmap43.1}}
$$

Diagram fibmap44. Maps and spaces for associated topological tangent vector and
vector-tuple bundles.
$$
\centerline{\epsfbox{fibmap44.1}}
$$

Diagram fibmap45. Maps and spaces for associated topological tangent vector and
vector-frame bundles.
$$
\centerline{\epsfbox{fibmap45.1}}
$$

% - - - - - - - - - - - - - - - - - - - - - - - - - - - - - - - - - - - - - - -
\filleject

Diagram fibmap46. Identity cross-sections for principal bundles.
$$
\centerline{\epsfbox{fibmap46.1}}
$$

Diagram fibmap47. Identity charts for principal bundles.
$$
\centerline{\epsfbox{fibmap47.1}}
$$

Diagram fibmap48. Connection form \localis/ation via local principal bundle
cross-sections.
$$
\centerline{\epsfbox{fibmap48.1}}
$$

Diagram fibmap49. Connection form \localis/ation component function.
$$
\centerline{\epsfbox{fibmap49.1}}
$$

% - - - - - - - - - - - - - - - - - - - - - - - - - - - - - - - - - - - - - - -
\filleject

Diagram fibmap50. Short-cut cross-sections for form-style non-topological
fibrations.
$$
\centerline{\epsfbox{fibmap50.1}}
$$

Diagram fibmap51. Form-style non-topological fibrations.
$$
\centerline{\epsfbox{fibmap51.1}}
$$

%==============================================================================
\secteject
\edef\SECTfibre{\the\pageno}\LinkName{fibre}

Diagram fibre1. \Fibre/ bundle map for the M\"obius strip.
$$
\centerline{\epsfbox{fibre1.1}}
$$

Diagram fibre2. \Fibre/ bundle chart for the M\"obius strip.
$$
\centerline{\epsfbox{fibre2.1}}
$$

Diagram fibre3. Two \fibre/ bundle charts for the M\"obius strip.
$$
\centerline{\epsfbox{fibre3.1}}
$$

% - - - - - - - - - - - - - - - - - - - - - - - - - - - - - - - - - - - - - - -
\filleject

Diagram fibre4. Direction-dependent parallelism example.
$$
\centerline{\epsfbox{fibre4.1}}
$$

Diagram fibre5. Direction-dependent connection example.
$$
\centerline{\epsfbox{fibre5.1}}
$$

Diagram fibre6. Map for the collage of three charts of a \fibre/ space.
$$
\centerline{\epsfbox{fibre6.1}}
$$

Diagram fibre7. Partitioning of a set by an inverse function.
$$
\centerline{\epsfbox{fibre7.1}}
$$

% - - - - - - - - - - - - - - - - - - - - - - - - - - - - - - - - - - - - - - -
\filleject

Diagram fibre8. Partitioning of a set by an inverse function. Variant of fibre7.
$$
\centerline{\epsfbox{fibre8.1}}
$$

Diagram fibre9. Partition of total space of \fibre/ bundle into \fibre/s.
$$
\centerline{\epsfbox{fibre9.1}}
$$

Diagram fibre10. Continuity of local chart.
$$
\centerline{\epsfbox{fibre10.1}}
$$

Diagram fibre11. Map for construction of a set collage.
$$
\centerline{\epsfbox{fibre11.1}}
$$

% - - - - - - - - - - - - - - - - - - - - - - - - - - - - - - - - - - - - - - -
\filleject

Diagram fibre12. Map for construction of a set collage, showing glued set.
$$
\centerline{\epsfbox{fibre12.1}}
$$

Diagram fibre13. \Fibre/ chart for \fibre/ bundle.
$$
\centerline{\epsfbox{fibre13.1}}
$$

Diagram fibre14. \Fibre/ bundle with absolute parallelism.
$$
\centerline{\epsfbox{fibre14.1}}
$$

Diagram fibre15. \Fibre/ bundle with pathwise parallelism.
$$
\centerline{\epsfbox{fibre15.1}}
$$

% - - - - - - - - - - - - - - - - - - - - - - - - - - - - - - - - - - - - - - -
\filleject

Diagram fibre16. Continuity of section of topological fibration. Based on
fibre10.
$$
\centerline{\epsfbox{fibre16.1}}
$$

Diagram fibre17. Porting parallelism between associated topological \fibre/
bundles.
$$
\centerline{\epsfbox{fibre17.1}}
$$

Diagrams fibre18. Parallelism on topological \fibre/ bundles.
$$
\centerline{\epsfbox{fibre18.1}}
$$

Diagrams fibre19. Parallelism on topological \fibre/ bundles. \Colour/ed version
of fibre18.mp.
$$
\centerline{\epsfbox{fibre19.1}}
$$

% - - - - - - - - - - - - - - - - - - - - - - - - - - - - - - - - - - - - - - -
\filleject

Diagram fibre20. \Fibre/ set isomorphisms through the charts. Based on fibre14.
$$
\centerline{\epsfbox{fibre20.1}}
$$

Diagram fibre21. \Fibre/ set automorphisms through the charts. Based on fibre20.
$$
\centerline{\epsfbox{fibre21.1}}
$$

Diagram fibre22. Topological parallelism structure.
$$
\centerline{\epsfbox{fibre22.1}}
$$

Diagram fibre23. Deviation from parallelism in \fibre/ bundle.
$$
\centerline{\epsfbox{fibre23.1}}
$$

% - - - - - - - - - - - - - - - - - - - - - - - - - - - - - - - - - - - - - - -
\filleject

Diagram fibre24. Curvature of connection on an ordinary \fibre/ bundle.
$$
\centerline{\epsfbox{fibre24.1}}
$$

Diagram fibre25. \Fibre/ bundle with parallelism bijection.
$$
\centerline{\epsfbox{fibre25.1}}
$$

Diagram fibre26. Parallelism lift function.
$$
\centerline{\epsfbox{fibre26.1}}
$$

% - - - - - - - - - - - - - - - - - - - - - - - - - - - - - - - - - - - - - - -
\filleject

Diagram fibre27. Parallel transport by an affine connection.
$$
\centerline{\epsfbox{fibre27.1}}
$$

Diagram fibre28. Estimating Riemann curvature by parallel transport around a
loop.
$$
\centerline{\epsfbox{fibre28.1}}
$$

% - - - - - - - - - - - - - - - - - - - - - - - - - - - - - - - - - - - - - - -
\filleject

Diagram fibre29. Defining Riemann curvature of a horizontal lift function using
a total-space curve-family.
$$
\centerline{\epsfbox{fibre29.1}}
$$

Diagram fibre30. Riemann curvature of parallel transport is a kind of ``exterior
derivative''.
$$
\centerline{\epsfbox{fibre30.1}}
$$

%==============================================================================
\secteject
\edef\SECTfield{\the\pageno}\LinkName{field}

Diagram field1. DG level 0. Points without attributes.
$$
\centerline{\epsfbox{field1.1}}
$$

Diagram field2. DG level 2. Charts and differentiable structure.
$$
\centerline{\epsfbox{field2.1}}
$$

Diagram field3. DG level 3. Parallel transport.
$$
\centerline{\epsfbox{field3.1}}
$$

Diagram field4. DG level 4. Construction of geodesics with a Riemannian metric.
$$
\centerline{\epsfbox{field4.1}}
$$

% - - - - - - - - - - - - - - - - - - - - - - - - - - - - - - - - - - - - - - -
\filleject

Diagram field5. DG level 4. Global Riemannian metric.
$$
\centerline{\epsfbox{field5.1}}
$$

Diagram field6. DG level 2. Charts and differentiable structure. Affine version.
$$
\centerline{\epsfbox{field6.1}}
$$

%==============================================================================
\secteject
\edef\SECTfn{\the\pageno}\LinkName{fn}

Diagram fn1. Step functions.
$$
\centerline{\epsfbox{fn1.1}}
$$

Diagram fn2. Functions $\arctan(x)$, $\arcsin(x)$, $\arccos(x)$.
$$
\centerline{\epsfbox{fn2.1}}
$$

Diagram fn3. Function $g:\reals\to\reals$ with $g:t\mapsto t\exp(-t^2)$.
$$
\centerline{\epsfbox{fn3.1}}
$$

Diagram fn4. Function $g:\reals\to\reals$ with $g:t\mapsto t\exp((1-t^2)/2)$.
$$
\centerline{\epsfbox{fn4.1}}
$$

% - - - - - - - - - - - - - - - - - - - - - - - - - - - - - - - - - - - - - - -
\filleject

Diagram fn6. A $C^\infty$ function in the range $[0,1]$.
$$
\centerline{\epsfbox{fn6.1}}
$$

Diagram fn15. $C^\infty$ function on $\reals^n$ with compact support~$B_R(x_0)$.
$$
\centerline{\epsfbox{fn15.1}}
$$

Diagram fn16. Modified from fn6. Two $C^\infty$ functions which are constant
outside~$[0,1]$.
$$
\centerline{\epsfbox{fn16.1}}
$$

Diagram fn17. A $C^\infty$ function which is zero outside~$B_{0,R}$,
$r=1$, $R=2$.
$$
\centerline{\epsfbox{fn17.1}}
$$

% - - - - - - - - - - - - - - - - - - - - - - - - - - - - - - - - - - - - - - -
\filleject

Diagram fn18. The functions $f(x)=\exp(-x^{-1})$ and
$g_R(x)=\exp((x^2-R^2)^{-1})$, $R=1$.
$$
\centerline{\epsfbox{fn18.1}}
$$

Diagram fn19. Left translation operators for Lie groups.
$$
\centerline{\epsfbox{fn19.1}}
$$

Diagram fn20. Right translation operators for Lie groups.
$$
\centerline{\epsfbox{fn20.1}}
$$

% - - - - - - - - - - - - - - - - - - - - - - - - - - - - - - - - - - - - - - -
\filleject

Diagram fn28. Subset of $\reals^2$ which is connected but not locally connected.
$$
\centerline{\epsfbox{fn28.1}}
$$

Diagram fn31. A $C^{0,1}$ transition map for an embedded manifold.
$$
\centerline{\epsfbox{fn31.1}}
$$

Diagram fn32. Projection maps for a $C^{0,1}$ manifold embedded
in~$\reals^{n+1}$.
$$
\centerline{\epsfbox{fn32.1}}
$$

% - - - - - - - - - - - - - - - - - - - - - - - - - - - - - - - - - - - - - - -
\filleject

Diagram fn35. A Lipschitz function with no one-sided derivatives at the origin.
$$
\centerline{\epsfbox{fn35.1}}
$$

Diagram fn36. Fractional powers $f(x)=\vert x\vert^\alpha$.
$$
\centerline{\epsfbox{fn36.1}}
$$

Diagram fn37. Definition of derivative of real-valued function of real variable.
$$
\centerline{\epsfbox{fn37.1}}
$$

% - - - - - - - - - - - - - - - - - - - - - - - - - - - - - - - - - - - - - - -
\filleject

Diagram fn38. Absolute value and sign (signum) functions.
$$
\centerline{\epsfbox{fn38.1}}
$$

Diagram fn39. Gradient of a function (or a Viking opera house).
$$
\centerline{\epsfbox{fn39.1}}
$$

Diagram fn44. Modulo (``relaxation oscillator'') functions.
$$
\centerline{\epsfbox{fn44.1}}
$$

Diagram fn45. Sawtooth functions.
$$
\centerline{\epsfbox{fn45.1}}
$$

Diagram fn46. Square functions.
$$
\centerline{\epsfbox{fn46.1}}
$$

% - - - - - - - - - - - - - - - - - - - - - - - - - - - - - - - - - - - - - - -
\filleject

Diagram fn47. Heaviside function multiplied by monomial.
$$
\centerline{\epsfbox{fn47.1}}
$$

Diagram fn48. Higher derivatives of functions may be partially defined
functions.
$$
\centerline{\epsfbox{fn48.1}}
$$

Diagram fn49. Function $u(x)=\exp(-x^{-2}/2)$ with $u''(x)+b(x)u'(x)=0$.
$$
\centerline{\epsfbox{fn49.1}}
$$

Diagram fn50. Function $u(x)=\vert x\vert^k$ for $k=2.1$.
$$
\centerline{\epsfbox{fn50.1}}
$$

% - - - - - - - - - - - - - - - - - - - - - - - - - - - - - - - - - - - - - - -
\filleject

Diagram fn54. Cone-shaped \neighbour/hood for defining the derivative of a real
function.
$$
\centerline{\epsfbox{fn54.1}}
$$

Diagram fn55. Definition of right derivative of real-valued function of real
variable.
$$
\centerline{\epsfbox{fn55.1}}
$$

Diagram fn56. Fractional-part and rounding functions.
$$
\centerline{\epsfbox{fn56.1}}
$$

Diagram fn59. Bases for free linear space on $\intsge$ over~$\reals$.
$$
\centerline{\epsfbox{fn59.1}}
$$

% - - - - - - - - - - - - - - - - - - - - - - - - - - - - - - - - - - - - - - -
\filleject

Diagram fn60. Component maps for free linear space on $\intsge$ over~$\reals$.
$$
\centerline{\epsfbox{fn60.1}}
$$

Diagram fn61. Zooming in on a differentiable function.
$$
\centerline{\epsfbox{fn61.1}}
$$

Diagram fn68. Joining two points in~$\reals^2$.
$$
\centerline{\epsfbox{fn68.1}}
$$

Diagram fn69. Joining two points in~$\ints^2$.
$$
\centerline{\epsfbox{fn69.1}}
$$

% - - - - - - - - - - - - - - - - - - - - - - - - - - - - - - - - - - - - - - -
\filleject

Diagram fn70. Noise on the real line.
$$
\centerline{\epsfbox{fn70.1}}
$$

Diagram fn71. Bound on differential quotient for a noise function.
$$
\centerline{\epsfbox{fn71.1}}
$$

Diagram fn72. Bound on alternative differential quotient for a noise function.
$$
\centerline{\epsfbox{fn72.1}}
$$

Diagram fn73. Letterbox bounds for real-valued function of real variable.
$$
\centerline{\epsfbox{fn73.1}}
$$

Diagram fn74. Letterbox bounds for real-valued function of real variable.
Higher resolution.
$$
\centerline{\epsfbox{fn74.1}}
$$

% - - - - - - - - - - - - - - - - - - - - - - - - - - - - - - - - - - - - - - -
\filleject

Diagram fn75. Letterbox parameters $(u,v)$.
$$
\centerline{\epsfbox{fn75.1}}
$$

Diagram fn76. Function with divergent letterbox aspect ratio.
$$
\centerline{\epsfbox{fn76.1}}
$$

Diagram fn77. Discontinuous function satisfying $\forall
S\in\powerset(Y),\,\Int(S)\subseteq f(\Int(f^{-1}(S)))$.
$$
\centerline{\epsfbox{fn77.1}}
$$

Diagram fn78. Function $f:\reals\to\reals$ with
$\liminf_{x\to0}f(x)<\limsup_{x\to0}f(x)$.
$$
\centerline{\epsfbox{fn78.1}}
$$

% - - - - - - - - - - - - - - - - - - - - - - - - - - - - - - - - - - - - - - -
\filleject

Diagram fn79. Geodesics in terrestrial coordinates on the two-sphere.
$$
\centerline{\epsfbox{fn79.1}}
$$

Diagram fn80. Definition of derivative of real-valued function of real variable.
Based on fn37.
$$
\centerline{\epsfbox{fn80.1}}
$$

Diagram fn81. Some integral and fractional powers of~$x\in\reals$. (An
accidentally interesting-looking diagram.)
$$
\centerline{\epsfbox{fn81.1}}
$$

% - - - - - - - - - - - - - - - - - - - - - - - - - - - - - - - - - - - - - - -
\filleject

Diagram fn82. Integral powers of~$x\in\reals$.
$$
\centerline{\epsfbox{fn82.1}}
$$

Diagram fn84. Explicit $\delta(\varepsilon)$ for given~$\varepsilon$.
$$
\centerline{\epsfbox{fn84.1}}
$$

Diagram fn85. Definition of Dini derivatives of real-valued function of real
variable.
$$
\centerline{\epsfbox{fn85.1}}
$$

% - - - - - - - - - - - - - - - - - - - - - - - - - - - - - - - - - - - - - - -
\filleject

Diagram fn86. Shadow-set lemma for proving the Lebesgue differentiation theorem.
(Riesz 1932.)
$$
\centerline{\epsfbox{fn86.1}}
$$

Diagram fn87. Map $x\mapsto(1+x/(1+\vert x\vert))/2$ from $\rats$ to
$\rats(0,1)$.
$$
\centerline{\epsfbox{fn87.1}}
$$

Diagram fn90. Double shadow-set lemma for proving the Lebesgue differentiation
theorem. (Riesz 1932.)
$$
\centerline{\epsfbox{fn90.1}}
$$

Diagram fn91. Enumeration of gradient pairs for double shadow set procedure.
$$
\centerline{\epsfbox{fn91.1}}
$$

% - - - - - - - - - - - - - - - - - - - - - - - - - - - - - - - - - - - - - - -
\filleject

Diagram fn94. Functions $(\frac12(1+a^2))^{1/3}+2((1+a)/(1+a^2))^{1/3}$ and
$9a^{1/3}/(2+a)$.
$$
\centerline{\epsfbox{fn94.1}}
$$

Diagram fn96. Function which is forward-connected, but not weakly connected.
$$
\centerline{\epsfbox{fn96.1}}
$$

Diagram fn97. Function with infinitely many limit points at the origin.
$$
\centerline{\epsfbox{fn97.1}}
$$

% - - - - - - - - - - - - - - - - - - - - - - - - - - - - - - - - - - - - - - -
\filleject

Diagram fn98. Divergent function with singleton limit set.
$$
\centerline{\epsfbox{fn98.1}}
$$

Diagram fn99. Graph of continuous curve with infinitely many constant stretches.
$$
\centerline{\epsfbox{fn99.1}}
$$

Diagram fn100. Function which is very weakly Lipschitz continuous.
$$
\centerline{\epsfbox{fn100.1}}
$$

% - - - - - - - - - - - - - - - - - - - - - - - - - - - - - - - - - - - - - - -
\filleject

Diagram fn101. Rational powers of non-negative real numbers.
$$
\centerline{\epsfbox{fn101.1}}
$$

Diagram fn102. Lower bound for a seminorm~$\psi$.
$$
\centerline{\epsfbox{fn102.1}}
$$

Diagram fn103. Non-regular inclusion of submanifold in $\reals^2$.
$$
\centerline{\epsfbox{fn103.1}}
$$

% - - - - - - - - - - - - - - - - - - - - - - - - - - - - - - - - - - - - - - -
\filleject

Diagram fn104. A global $C^k$ extension of a locally constant function on a
$C^k$ manifold.
$$
\centerline{\epsfbox{fn104.1}}
$$

%==============================================================================
\secteject
\edef\SECTfnmap{\the\pageno}\LinkName{fnmap}

Diagram fnmap1. Composition of functions yields a partially defined function.
$$
\centerline{\epsfbox{fnmap1.1}}
$$

Diagram fnmap2. Multilinear map with 3 source spaces.
$$
\centerline{\epsfbox{fnmap2.1}}
$$

Diagram fnmap3. Trilinear map definition. Based on fn34.
$$
\centerline{\epsfbox{fnmap3.1}}
$$

Diagram fnmap4. Domain and range of composite of two local transformations.
$$
\centerline{\epsfbox{fnmap4.1}}
$$

% - - - - - - - - - - - - - - - - - - - - - - - - - - - - - - - - - - - - - - -
\filleject

Diagram fnmap5. \Generalis/ed right inverse function.
$$
\centerline{\epsfbox{fnmap5.1}}
$$

Diagram fnmap6. Set map between power sets.
$$
\centerline{\epsfbox{fnmap6.1}}
$$

%==============================================================================
\secteject
\edef\SECTfnTWOd{\the\pageno}\LinkName{fnTWOd}

Diagram fn2d1. Range of exponential map.
$$
\centerline{\epsfbox{fn2d1.1}}
$$

Diagram fn2d2. Ranges of exponential maps.
$$
\centerline{\epsfbox{fn2d2.1}}
$$

Diagram fn2d3. Ranges of exponential maps.
$$
\centerline{\epsfbox{fn2d3.1}}
$$

% - - - - - - - - - - - - - - - - - - - - - - - - - - - - - - - - - - - - - - -
\filleject

Diagram fn2d4. Range of terrestrial coordinates for $S^2$.
$$
\centerline{\epsfbox{fn2d4.1}}
$$

Diagram fn2d5. Gradient of a parabolic function.
$$
\centerline{\epsfbox{fn2d5.1}}
$$

Diagram fn2d6. Effect of diffeomorphism on the gradient of a quadratic function.
$$
\centerline{\epsfbox{fn2d6.1}}
$$

% - - - - - - - - - - - - - - - - - - - - - - - - - - - - - - - - - - - - - - -
\filleject

Diagram fn2d7. Real-valued multilinear function of two real variables.
$$
\centerline{\epsfbox{fn2d7.1}}
$$

Diagram fn2d8. Stokes theorem on a rectangle.
$$
\centerline{\epsfbox{fn2d8.1}}
$$

Diagram fn2d9. Level curves of real-valued function $f:\reals^2\to\reals$,
$f:x\mapsto2x_1x_2/(x_1^2+x_2^2)$ which is partially differentiable on
$\reals^2$, but not continuous at~$x=(0,0)$.
$$
\centerline{\epsfbox{fn2d9.1}}
$$

% - - - - - - - - - - - - - - - - - - - - - - - - - - - - - - - - - - - - - - -
\filleject

Diagram fn2d10. Level curves of real-valued function $f:\reals^2\to\reals$,
$f:x\mapsto x_1^2 x_1^{-1}\exp\bigl((1-x_2^{4}x_1^{-2})/2\bigr)$ which is
directionally differentiable on $\reals^2$, but not continuous at~$x=(0,0)$.
$$
\centerline{\epsfbox{fn2d10.1}}
$$

Diagram fn2d11. Level curves of real-valued function $f:\reals^2\to\reals$,
$f:x\mapsto g(x_1^2 x_1^{-1})$, $g:\reals\to\reals$, $g:t\mapsto
t\exp((1-t^2)/2)$ which is directionally differentiable on $\reals^2$, but not
continuous at~$x=(0,0)$.
$$
\centerline{\epsfbox{fn2d11.1}}
$$

Diagram fn2d12. Graph suggesting the existence of complex pole for
$(1+x^2)^{-1}$.
$$
\centerline{\epsfbox{fn2d12.1}}
$$

% - - - - - - - - - - - - - - - - - - - - - - - - - - - - - - - - - - - - - - -
\filleject

Diagram fn2d13. Chart for Lipschitz manifold. Non-differentiable on the axes.
$$
\centerline{\epsfbox{fn2d13.1}}
$$

Diagram fn2d14. Chart for Lipschitz manifold. Non-differentiable at the origin.
$$
\centerline{\epsfbox{fn2d14.1}}
$$

Diagram fn2d15. Chart for Lipschitz manifold. Non-differentiable at the origin.
$$
\centerline{\epsfbox{fn2d15.1}}
$$

% - - - - - - - - - - - - - - - - - - - - - - - - - - - - - - - - - - - - - - -
\filleject

Diagram fn2d16. Distinction between restricted point-maps and restricted-set
maps.
$$
\centerline{\epsfbox{fn2d16.1}}
$$

Diagram fn2d17. Geodesic curves for a slightly non-Riemannian manifold:
$g_{ij}(x)=9x_1^4\delta_{i1}\delta_{j1}+\delta_{i2}\delta_{j2}$.
$$
\centerline{\epsfbox{fn2d17.1}}
$$

Diagram fn2d18. Graph of Picard iterants for uniform orthogonal translation.
$$
\centerline{\epsfbox{fn2d18.1}}
$$

% - - - - - - - - - - - - - - - - - - - - - - - - - - - - - - - - - - - - - - -
\filleject

Diagram fn2d19. Topological space which is connected, but not locally connected.
$$
\centerline{\epsfbox{fn2d19.1}}
$$

Diagram fn2d20. Topological space which is neither connected nor locally
connected.
$$
\centerline{\epsfbox{fn2d20.1}}
$$

Diagram fn2d21. Graph of Peano space-filling curve.
$$
\centerline{\epsfbox{fn2d21.1}}
$$

% - - - - - - - - - - - - - - - - - - - - - - - - - - - - - - - - - - - - - - -
\filleject

Diagram fn2d22. $f:(0,2\pi)\to\reals^2$ with $f:t\mapsto(\sin t,\sin 2t)$. Not a
topological manifold embedding.
$$
\centerline{\epsfbox{fn2d22.1}}
$$

Diagram fn2d23. Level curves of function which is not totally differentiable.
$$
\centerline{\epsfbox{fn2d23.1}}
$$

Diagram fn2d24. Unit level curves of $p$-norms for~$\reals^2$.
$$
\centerline{\epsfbox{fn2d24.1}}
$$

% - - - - - - - - - - - - - - - - - - - - - - - - - - - - - - - - - - - - - - -
\filleject

Diagram fn2d25. Unit level curves of norms $x\mapsto\max(k\vert
x_1+x_2\vert,x_1-x_2\vert)$ for~$\reals^2$.
$$
\centerline{\epsfbox{fn2d25.1}}
$$

Diagram fn2d26. Integral curves of the vector field $(x,y)\mapsto(x+y,y-x)$.
$$
\centerline{\epsfbox{fn2d26.1}}
$$

%==============================================================================
\secteject
\edef\SECTgrid{\the\pageno}\LinkName{grid}

Diagram grid1. Testing the {\tt G\_lingrid} macro for linear grid interpolation.
$$
\centerline{\epsfbox{grid1.1}}
$$

Diagram grid2. Testing the {\tt G\_grid} macro for curvilinear grid
interpolation.
$$
\centerline{\epsfbox{grid2.1}}
$$

Diagram grid3. Testing the {\tt G\_grid} macro for curvilinear grid
interpolation.
$$
\centerline{\epsfbox{grid3.1}}
$$

% - - - - - - - - - - - - - - - - - - - - - - - - - - - - - - - - - - - - - - -
\filleject

Diagram grid4. A family of curves.
$$
\centerline{\epsfbox{grid4.1}}
$$

Diagram grid5. A vector field on a manifold.
$$
\centerline{\epsfbox{grid5.1}}
$$

Diagram grid6. A vector field inside a manifold.
$$
\centerline{\epsfbox{grid6.1}}
$$

Diagram grid7. Parallel transport of a vector over a family of curves.
$$
\centerline{\epsfbox{grid7.1}}
$$

% - - - - - - - - - - - - - - - - - - - - - - - - - - - - - - - - - - - - - - -
\filleject

Diagram grid8. Parallel transport of a vector over a family of curves.
$$
\centerline{\epsfbox{grid8.1}}
$$

Diagram grid9. Interpretation of Lie derivative using flow of curve $\beta$ with
specified velocity~$V$.
$$
\centerline{\epsfbox{grid9.1}}
$$

%==============================================================================
\secteject
\edef\SECTint{\the\pageno}\LinkName{int}

Diagram int1. The first seven ordinal numbers.
$$
\centerline{\epsfbox{int1.1}}
$$

Diagram int2. The ordinal number 10, compact version. Based on int5.
$$
\centerline{\epsfbox{int2.1}}
$$

Diagram int3. Iteration of universe sets.
$$
\centerline{\epsfbox{int3.1}}
$$

% - - - - - - - - - - - - - - - - - - - - - - - - - - - - - - - - - - - - - - -
\filleject

Diagram int4. The ordinal number 10.
$$
\centerline{\epsfbox{int4.1}}
$$

% - - - - - - - - - - - - - - - - - - - - - - - - - - - - - - - - - - - - - - -
\filleject

Diagram int5. The ordinal number 12.
$$
\centerline{\epsfbox{int5.1}}
$$

% - - - - - - - - - - - - - - - - - - - - - - - - - - - - - - - - - - - - - - -
\filleject

Diagram int6. Proof strategy to show total order of finite ordinals.
$$
\centerline{\epsfbox{int6.1}}
$$

Diagram int7. The first five von Neumann universes. (Actually published by
Zermelo in 1930.)
$$
\centerline{\epsfbox{int7.1}}
$$

Diagram int9. Enumeration of an arbitrary subset of~$\omega$.
$$
\centerline{\epsfbox{int9.1}}
$$

Diagram int10. Proof that an injective finite set endomorphism is a bijection.
$$
\centerline{\epsfbox{int10.1}}
$$

% - - - - - - - - - - - - - - - - - - - - - - - - - - - - - - - - - - - - - - -
\filleject

Diagram int11. Construction of rational numbers from integers.
$$
\centerline{\epsfbox{int11.1}}
$$

Diagram int12. Construction of signed integers from pairs of unsigned integers.
$$
\centerline{\epsfbox{int12.1}}
$$

Diagram int13. Definition of signed integers in terms of integers and a ``sign
bit''.
$$
\centerline{\epsfbox{int13.1}}
$$

Diagram int14. Addition of finite ordinal numbers.
$$
\centerline{\epsfbox{int14.1}}
$$

% - - - - - - - - - - - - - - - - - - - - - - - - - - - - - - - - - - - - - - -
\filleject

Diagram int15. Application of ZF regularity (well-foundedness) to an ordinal
number.
$$
\centerline{\epsfbox{int15.1}}
$$

%==============================================================================
\secteject
\edef\SECTire{\the\pageno}\LinkName{ire}

Diagram ire1. Topology of the counties of Ireland.
$$
\centerline{\epsfbox{ire1.1}}
$$

Diagram ire2. Topology of the counties of Ireland using atlas coordinates.
$$
\centerline{\epsfbox{ire2.1}}
$$

% - - - - - - - - - - - - - - - - - - - - - - - - - - - - - - - - - - - - - - -
\filleject

Diagram ire3. Topology of the counties of Ireland in Irish language.
$$
\centerline{\epsfbox{ire3.1}}
$$

%==============================================================================
\secteject
\edef\SECTlinmap{\the\pageno}\LinkName{linmap}

Diagram linmap1. Maps for duals of tensor spaces.
$$
\centerline{\epsfbox{linmap1.1}}
$$

Diagram linmap2. Conversion of linear map to matrix multiplication.
$$
\centerline{\epsfbox{linmap2.1}}
$$

Diagram linmap3. Conversion of bilinear map to matrix multiplication.
$$
\centerline{\epsfbox{linmap3.1}}
$$

% - - - - - - - - - - - - - - - - - - - - - - - - - - - - - - - - - - - - - - -
\filleject

Diagram linmap4. Transpose map of a linear space monomorphism.
$$
\centerline{\epsfbox{linmap4.1}}
$$

Diagram linmap5. Component maps for linear spaces.
$$
\centerline{\epsfbox{linmap5.1}}
$$

Diagram linmap6. Transition maps between component maps for linear spaces.
$$
\centerline{\epsfbox{linmap6.1}}
$$

Diagram linmap7. Transition maps between component maps for linear spaces with
indexed bases.
$$
\centerline{\epsfbox{linmap7.1}}
$$

% - - - - - - - - - - - - - - - - - - - - - - - - - - - - - - - - - - - - - - -
\filleject

Diagram linmap8. Component transition maps for a linear map.
$$
\centerline{\epsfbox{linmap8.1}}
$$

Diagram linmap9. Maps and spaces for a linear space quotient.
$$
\centerline{\epsfbox{linmap9.1}}
$$

Diagram linmap10. Maps and spaces for an isomorphism for the dual of a subspace
of a linear space.
$$
\centerline{\epsfbox{linmap10.1}}
$$

Diagram linmap11. Maps and spaces for an isomorphism for the dual of a subspace
of a dual space.
$$
\centerline{\epsfbox{linmap11.1}}
$$

% - - - - - - - - - - - - - - - - - - - - - - - - - - - - - - - - - - - - - - -
\filleject

Diagram linmap12. Maps and spaces for an isomorphism for the dual of a subspace
of a tensor space.
$$
\centerline{\epsfbox{linmap12.1}}
$$

Diagram linmap13. Maps and spaces for an isomorphism for the dual of
antisymmetric multilinear functions.
$$
\centerline{\epsfbox{linmap13.1}}
$$

Diagram linmap14. Maps and spaces for a linear space quotient of the kernel of a
linear map.
$$
\centerline{\epsfbox{linmap14.1}}
$$

%==============================================================================
\secteject
\edef\SECTlog{\the\pageno}\LinkName{log}

Diagram log1. Reduction of geometry and arithmetic to symbolic logic, then
interpretation.
$$
\centerline{\epsfbox{log1.1}}
$$

Diagram log2. Processing of logical propositions into ``true'' and ``false''.
$$
\centerline{\epsfbox{log2.1}}
$$

Diagram log3. Processing of logical propositions into ``true'' and ``false''.
$$
\centerline{\epsfbox{log3.1}}
$$

Diagram log4. Equivalence for conjunction of propositions.
$$
\centerline{\epsfbox{log4.1}}
$$

Diagram log5. Equivalence for disjunction of propositions?
$$
\centerline{\epsfbox{log5.1}}
$$

% - - - - - - - - - - - - - - - - - - - - - - - - - - - - - - - - - - - - - - -
\filleject

Diagram log6. Equivalence for conjunction of propositions. Inversion of
log4.
$$
\centerline{\epsfbox{log6.1}}
$$

Diagram log7. Equivalence for disjunction of propositions? Inversion of
log5.
$$
\centerline{\epsfbox{log7.1}}
$$

Diagram log8. Object spaces for propositional calculus.
$$
\centerline{\epsfbox{log8.1}}
$$

Diagram log9. Object spaces and name spaces for propositional calculus. (See
log43.)
$$
\centerline{\epsfbox{log9.1}}
$$

% - - - - - - - - - - - - - - - - - - - - - - - - - - - - - - - - - - - - - - -
\filleject

Diagram log10. Meaning of propositions: description versus prescription.
$$
\centerline{\epsfbox{log10.1}}
$$

Diagram log11. Mapping meta-meta-propositions to meta-propositions, and then to
concrete propositions.
$$
\centerline{\epsfbox{log11.1}}
$$

Diagram log12. The relation of ideas to \verbalis/ation, both spoken and
written.
$$
\centerline{\epsfbox{log12.1}}
$$

Diagram log13. The relation of ideas to spoken \verbalis/ation.
$$
\centerline{\epsfbox{log13.1}}
$$

Diagram log14. Abstract logic in relation to concrete applications.
$$
\centerline{\epsfbox{log14.1}}
$$

% - - - - - - - - - - - - - - - - - - - - - - - - - - - - - - - - - - - - - - -
\filleject

Diagram log15. Decision-making for mammoth hunters.
$$
\centerline{\epsfbox{log15.1}}
$$

Diagram log16. Definition of proposition name space and concrete proposition
domain.
$$
\centerline{\epsfbox{log16.1}}
$$

Diagram log17. Concrete proposition domains and truth value maps.
$$
\centerline{\epsfbox{log17.1}}
$$

% - - - - - - - - - - - - - - - - - - - - - - - - - - - - - - - - - - - - - - -
\filleject

Diagram log18. Single proposition name space with multiple concrete proposition
domains.
$$
\centerline{\epsfbox{log18.1}}
$$

Diagram log19. Parsing a logical expression.
$$
\centerline{\epsfbox{log19.1}}
$$

Diagram log20. World-model ontology for logical negation.
$$
\centerline{\epsfbox{log20.1}}
$$

% - - - - - - - - - - - - - - - - - - - - - - - - - - - - - - - - - - - - - - -
\filleject

Diagram log21. Abstract/concrete predicates, variables and propositions.
$$
\centerline{\epsfbox{log21.1}}
$$

Diagram log22. Ontology of unknown truth values.
$$
\centerline{\epsfbox{log22.1}}
$$

Diagram log23. Ontology of unknown truth values. Two modelled logic machines.
$$
\centerline{\epsfbox{log23.1}}
$$

% - - - - - - - - - - - - - - - - - - - - - - - - - - - - - - - - - - - - - - -
\filleject

Diagram log24. Ontology of unknown truth values. Second-order modelling of
unknowns.
$$
\centerline{\epsfbox{log24.1}}
$$

Diagram log25. Modelling of modelling.
$$
\centerline{\epsfbox{log25.1}}
$$

Diagram log26. Meta-modelling within one machine (or mind).
$$
\centerline{\epsfbox{log26.1}}
$$

Diagram log27. Recursive modelling of a machine by a second machine.
$$
\centerline{\epsfbox{log27.1}}
$$

% - - - - - - - - - - - - - - - - - - - - - - - - - - - - - - - - - - - - - - -
\filleject

Diagram log28. Recursive modelling within one mind.
$$
\centerline{\epsfbox{log28.1}}
$$

Diagram log29. Simulation of a computer simulating a computer.
$$
\centerline{\epsfbox{log29.1}}
$$

Diagram log30. Classes and objects in multiple machines.
$$
\centerline{\epsfbox{log30.1}}
$$

Diagram log31. Classes and objects in multiple machines. Recursive.
$$
\centerline{\epsfbox{log31.1}}
$$

Diagram log32. Classes and objects in multiple machines. Unbounded class nesting
depth.
$$
\centerline{\epsfbox{log32.1}}
$$

% - - - - - - - - - - - - - - - - - - - - - - - - - - - - - - - - - - - - - - -
\filleject

Diagram log33. Parsing a quantified logic expression.
$$
\centerline{\epsfbox{log33.1}}
$$

Diagram log34. Set membership tree for the set of finite ordinal numbers.
$$
\centerline{\epsfbox{log34.1}}
$$

Diagram log35. Real world, probabilistic model, logical model and current
knowledge.
$$
\centerline{\epsfbox{log35.1}}
$$

Diagram log36. The analysis-synthesis reductionism loop.
$$
\centerline{\epsfbox{log36.1}}
$$

% - - - - - - - - - - - - - - - - - - - - - - - - - - - - - - - - - - - - - - -
\filleject

Diagram log37. Open subbase for a topology.
$$
\centerline{\epsfbox{log37.1}}
$$

Diagram log38. Minimal and maximal elements for a partially ordered set.
$$
\centerline{\epsfbox{log38.1}}
$$

Diagram log39. Arrow diagram for a totally ordered set.
$$
\centerline{\epsfbox{log39.1}}
$$

Diagram log40. Inclusion relation on power set. Partial orders which are not
total orders.
$$
\centerline{\epsfbox{log40.1}}
$$

Diagram log41. Inclusion relation on power set of four letters.
$$
\centerline{\epsfbox{log41.1}}
$$

% - - - - - - - - - - - - - - - - - - - - - - - - - - - - - - - - - - - - - - -
\filleject

Diagram log42. Different ways of viewing the logical operation $A\imprel B$.
$$
\centerline{\epsfbox{log42.1}}
$$

Diagram log43. Object spaces and name spaces for propositional calculus. (Based
on log9.)
$$
\centerline{\epsfbox{log43.1}}
$$

Diagram log44. Representations of an example logical expression.
$$
\centerline{\epsfbox{log44.1}}
$$

% - - - - - - - - - - - - - - - - - - - - - - - - - - - - - - - - - - - - - - -
\filleject

Diagram log45. Bracket-level diagram for an infix logical expression.
$$
\centerline{\epsfbox{log45.1}}
$$

Diagram log46. Possible origin of single-model and multi-model logical
languages.
$$
\centerline{\epsfbox{log46.1}}
$$

Diagram log47. The ``shim'' concept for connecting ZF+AC theorems to ZF
theorems.
$$
\centerline{\epsfbox{log47.1}}
$$

Diagram log48. Classes of cardinality of finite, mediate and infinite sets.
$$
\centerline{\epsfbox{log48.1}}
$$

% - - - - - - - - - - - - - - - - - - - - - - - - - - - - - - - - - - - - - - -
\filleject

Diagram log49. Eight independent definitions for a finite set.
$$
\centerline{\epsfbox{log49.1}}
$$

Diagram log50. Object spaces and name spaces for predicate calculus. (Based on
log9.)
$$
\centerline{\epsfbox{log50.1}}
$$

Diagram log51. Set membership tree for the set of finite ordinal numbers. (Based
on log34.)
$$
\centerline{\epsfbox{log51.1}}
$$

% - - - - - - - - - - - - - - - - - - - - - - - - - - - - - - - - - - - - - - -
\filleject

Diagram log52. Inclusion relation on power set of two, three or four letters.
(See log40 and log41.)
$$
\centerline{\epsfbox{log52.1}}
$$

%==============================================================================
\secteject
\edef\SECTlogmap{\the\pageno}\LinkName{logmap}

Diagram logmap1. Contradictory logic.
$$
\centerline{\epsfbox{logmap1.1}}
$$

Diagram logmap2. Commutative diagram for socio-mathematical \synchronis/ation.
$$
\centerline{\epsfbox{logmap2.1}}
$$

Diagram logmap3. Sensor/motor loop between mind and matter.
$$
\centerline{\epsfbox{logmap3.1}}
$$

Diagram logmap4. Correspondence between concrete relations and abstract
relations.
$$
\centerline{\epsfbox{logmap4.1}}
$$

% - - - - - - - - - - - - - - - - - - - - - - - - - - - - - - - - - - - - - - -
\filleject

Diagram logmap5. Abstract and concrete equality relations.
$$
\centerline{\epsfbox{logmap5.1}}
$$

Diagram logmap6. The naming bottleneck and dark numbers.
$$
\centerline{\epsfbox{logmap6.1}}
$$

Diagram logmap7. Constant names for fixed concrete objects.
$$
\centerline{\epsfbox{logmap7.1}}
$$

Diagram logmap8. The naming bottleneck and dark numbers.
$$
\centerline{\epsfbox{logmap8.1}}
$$

% - - - - - - - - - - - - - - - - - - - - - - - - - - - - - - - - - - - - - - -
\filleject

Diagram logmap9. Real numbers exist only in observers, not in the observed
physical systems.
$$
\centerline{\epsfbox{logmap9.1}}
$$

%==============================================================================
\secteject
\edef\SECTmap{\the\pageno}\LinkName{map}

Diagram map5. The intersection of two subsets of a set~$S$.
$$
\centerline{\epsfbox{map5.1}}
$$

Diagram map85. Example of transformation group homomorphism.
$$
\centerline{\epsfbox{map85.1}}
$$

Diagram map86. Maps/spaces for a transformation group homomorphism.
$$
\centerline{\epsfbox{map86.1}}
$$

% - - - - - - - - - - - - - - - - - - - - - - - - - - - - - - - - - - - - - - -
% \filleject

Diagram map103. Group conjugation interpreted as parallel transport.
$$
\centerline{\epsfbox{map103.1}}
$$

Diagram map145. Relation between real geometry, Euclidean geometry and Cartesian
geometry.
$$
\centerline{\epsfbox{map145.1}}
$$

%==============================================================================
\secteject
\edef\SECTmat{\the\pageno}\LinkName{mat}

Diagram mat1. Component transition matrices for particular bases of a linear
space.
$$
\centerline{\epsfbox{mat1.1}}
$$

Diagram mat2. Two enumeration styles for the set $\omega\times\omega$.
$$
\centerline{\epsfbox{mat2.1}}
$$

%==============================================================================
\secteject
\edef\SECTmetric{\the\pageno}\LinkName{metric}

Diagram metric1. Radii of open balls within the intersection of two given balls.
$$
\centerline{\epsfbox{metric1.1}}
$$

Diagram metric2. Open and closed balls in a metric space.
$$
\centerline{\epsfbox{metric2.1}}
$$

Diagram metric3. Equivalents to triangle inequality. Outer bounds.
$$
\centerline{\epsfbox{metric3.1}}
$$

Diagram metric4. Equivalents to triangle inequality. Inner bounds.
$$
\centerline{\epsfbox{metric4.1}}
$$

% - - - - - - - - - - - - - - - - - - - - - - - - - - - - - - - - - - - - - - -
\filleject

Diagram metric5. Distance function on a torus $[0,4)\times[0,3)$.
$$
\centerline{\epsfbox{metric5.1}}
$$

Diagram metric6. Pseudo-distance function on a torus $[0,4)\times[0,3)$.
$$
\centerline{\epsfbox{metric6.1}}
$$

Diagram metric7. Pseudo-distance function on~$\reals^2$.
$$
\centerline{\epsfbox{metric7.1}}
$$

%==============================================================================
\secteject
\edef\SECTmisc{\the\pageno}\LinkName{misc}

Diagram misc1. The \parametris/ation of the circle.
$$
\centerline{\epsfbox{misc1.1}}
$$

Diagram misc2. Processes in writing a book on the net.
$$
\centerline{\epsfbox{misc2.1}}
$$

Diagram misc3. Processes in writing a book on the net. Adapted from misc2.
$$
\centerline{\epsfbox{misc3.1}}
$$

Diagram misc4. Cycle of dependencies of disciplines.
$$
\centerline{\epsfbox{misc4.1}}
$$

% - - - - - - - - - - - - - - - - - - - - - - - - - - - - - - - - - - - - - - -
\filleject

Diagram misc5. MSC\ts 2000 subject list.
$$
\centerline{\epsfbox{misc5.1}}
$$

% - - - - - - - - - - - - - - - - - - - - - - - - - - - - - - - - - - - - - - -
\filleject

Diagram misc7. MSC\ts 2000 subject list with extra labels.
$$
\centerline{\epsfbox{misc7.1}}
$$

% - - - - - - - - - - - - - - - - - - - - - - - - - - - - - - - - - - - - - - -
\filleject

Diagram misc6. A rectangle with dotted line.
$$
\centerline{\epsfbox{misc6.1}}
$$

Diagram misc14. Proof of the Schr\"oder-Bernstein theorem.
$$
\centerline{\epsfbox{misc14.1}}
$$

Diagram misc15. Cross product of two sets.
$$
\centerline{\epsfbox{misc15.1}}
$$

% - - - - - - - - - - - - - - - - - - - - - - - - - - - - - - - - - - - - - - -
\filleject

Diagram misc16. Cycle of dependencies of disciplines. Based on misc4.
$$
\centerline{\epsfbox{misc16.1}}
$$

Diagram misc18. Classification of entities in the environment.
$$
\centerline{\epsfbox{misc18.1}}
$$

Diagram misc20. The non-applicability of truth tables to predicate logic.
$$
\centerline{\epsfbox{misc20.1}}
$$

% - - - - - - - - - - - - - - - - - - - - - - - - - - - - - - - - - - - - - - -
\filleject

Diagram misc19. MSC\ts 2010 subject list with two labels and Australian
spelling.
$$
\centerline{\epsfbox{misc19.1}}
$$

Diagram misc21. Finite, infinite, Dedekind-finite and Dedekind-infinite sets.
$$
\centerline{\epsfbox{misc21.1}}
$$

% - - - - - - - - - - - - - - - - - - - - - - - - - - - - - - - - - - - - - - -
\filleject

Diagram misc22. MSC\ts 2010 subject list with Australian spelling and clockwise
orientation.
$$
\centerline{\epsfbox{misc22.1}}
$$

Diagram misc23. Crazy order of week-day names.
Day-name$[n]={}$Planet-name$[24n\bmod 7]$, $n=0\dots6$.
$$
\centerline{\epsfbox{misc23.1}}
$$

%==============================================================================
\secteject
\edef\SECTprodmap{\the\pageno}\LinkName{prodmap}

Diagram prodmap1. Homeomorphism for subset for product topology.
$$
\centerline{\epsfbox{prodmap1.1}}
$$

Diagram prodmap2. Maps and spaces for the pointwise Cartesian product of two
functions.
$$
\centerline{\epsfbox{prodmap2.1}}
$$

Diagram prodmap3. Homeomorphism for subset for product topology. Adapted from
prodmap1.
$$
\centerline{\epsfbox{prodmap3.1}}
$$

Diagram prodmap4. Cartesian product of two topological spaces.
$$
\centerline{\epsfbox{prodmap4.1}}
$$

% - - - - - - - - - - - - - - - - - - - - - - - - - - - - - - - - - - - - - - -
\filleject

Diagram prodmap5. Homeomorphism for subset for product topology. Adapted from
prodmap3.
$$
\centerline{\epsfbox{prodmap5.1}}
$$

Diagram prodmap6. Direct product of diffeomorphisms between Cartesian space
regions.
$$
\centerline{\epsfbox{prodmap6.1}}
$$

Diagram prodmap7. Direct product of continuous maps between Cartesian space
regions.
$$
\centerline{\epsfbox{prodmap7.1}}
$$

% - - - - - - - - - - - - - - - - - - - - - - - - - - - - - - - - - - - - - - -
\filleject

Diagram prodmap8. Diffeomorphism between slice of set diffeomorphic to product
of sets.
$$
\centerline{\epsfbox{prodmap8.1}}
$$

Diagram prodmap9. The ``partial maps'' of $C^k$ map $\phi:M_1\times M_2\to M_0$.
$$
\centerline{\epsfbox{prodmap9.1}}
$$

%==============================================================================
\secteject
\edef\SECTreal{\the\pageno}\LinkName{real}

Diagram real1. Proof of Heine-Borel theorem.
$$
\centerline{\epsfbox{real1.1}}
$$

Diagram real2. Non-Hausdorff locally Cartesian space example.
$$
\centerline{\epsfbox{real2.1}}
$$

Diagram real3. Ambiguity of determination of curve by set and end-points.
$$
\centerline{\epsfbox{real3.1}}
$$

Diagram real4. Contrast between curve map and 1-manifold maps.
$$
\centerline{\epsfbox{real4.1}}
$$

% - - - - - - - - - - - - - - - - - - - - - - - - - - - - - - - - - - - - - - -
\filleject

Diagram real5. Inverse manifold atlas for a path.
$$
\centerline{\epsfbox{real5.1}}
$$

Diagram real9. Addition of vectors in the free linear space over the reals.
$$
\centerline{\epsfbox{real9.1}}
$$

Diagram real11. Concatenation of curves using domain interval translations.
$$
\centerline{\epsfbox{real11.1}}
$$

% - - - - - - - - - - - - - - - - - - - - - - - - - - - - - - - - - - - - - - -
\filleject

Diagram real12. Concatenation of curves using domain interval continuous
transformations.
$$
\centerline{\epsfbox{real12.1}}
$$

Diagram real13. Concatenation of curves to prove transitivity of pathwise
connectedness.
$$
\centerline{\epsfbox{real13.1}}
$$

Diagram real14. Showing that the rationals have measure zero.
$$
\centerline{\epsfbox{real14.1}}
$$

Diagram real15. The real line with two origins.
$$
\centerline{\epsfbox{real15.1}}
$$

% - - - - - - - - - - - - - - - - - - - - - - - - - - - - - - - - - - - - - - -
\filleject

Diagram real16. The real line with two closed unit intervals.
$$
\centerline{\epsfbox{real16.1}}
$$

Diagram real17. A double real line with four origins.
$$
\centerline{\epsfbox{real17.1}}
$$

Diagram real18. Topology of a double real line with four origins.
$$
\centerline{\epsfbox{real18.1}}
$$

Diagram real19. Uneven distribution of rational numbers with
denominator${}\le25$.
$$
\centerline{\epsfbox{real19.1}}
$$

Diagram real20. The ``stages'' of an enumeration of the rational numbers.
$$
\centerline{\epsfbox{real20.1}}
$$

Diagram real21. Refinement of an interval partition.
$$
\centerline{\epsfbox{real21.1}}
$$

%==============================================================================
\secteject
\edef\SECTtangmap{\the\pageno}\LinkName{tangmap}

Diagram tangmap1. Charts and projection maps for tangent vector and covector
spaces.
$$
\centerline{\epsfbox{tangmap1.1}}
$$

Diagram tangmap2. Charts for tangent space of a tangent space.
$$
\centerline{\epsfbox{tangmap2.1}}
$$

Diagram tangmap3. Maps and spaces for differentials, both vectors and operators.
$$
\centerline{\epsfbox{tangmap3.1}}
$$

Diagram tangmap4. Maps and spaces for differentials.
$$
\centerline{\epsfbox{tangmap4.1}}
$$

% - - - - - - - - - - - - - - - - - - - - - - - - - - - - - - - - - - - - - - -
\filleject

Diagram tangmap5. Tangent space maps.
$$
\centerline{\epsfbox{tangmap5.1}}
$$

Diagram tangmap6. Differential of a map between two manifolds.
$$
\centerline{\epsfbox{tangmap6.1}}
$$

Diagram tangmap7. Differential of a map between two manifolds.
$$
\centerline{\epsfbox{tangmap7.1}}
$$

Diagram tangmap8. Tangent bundle maps.
$$
\centerline{\epsfbox{tangmap8.1}}
$$

% - - - - - - - - - - - - - - - - - - - - - - - - - - - - - - - - - - - - - - -
\filleject

Diagram tangmap9. Spaces and maps for the tangent vector bundle.
$$
\centerline{\epsfbox{tangmap9.1}}
$$

Diagram tangmap10. Spaces and maps for a tangent bundle.
$$
\centerline{\epsfbox{tangmap10.1}}
$$

Diagram tangmap11. Spaces and maps for the total tangent $r$-frame space.
$$
\centerline{\epsfbox{tangmap11.1}}
$$

Diagram tangmap12. Differentiability of a map between manifolds.
$$
\centerline{\epsfbox{tangmap12.1}}
$$

% - - - - - - - - - - - - - - - - - - - - - - - - - - - - - - - - - - - - - - -
\filleject

Diagram tangmap13. Differentiability of a map between manifolds.
$$
\centerline{\epsfbox{tangmap13.1}}
$$

Diagram tangmap14. Differential of real-valued function for second-order tangent
operators.
$$
\centerline{\epsfbox{tangmap14.1}}
$$

Diagram tangmap15. Differential of a differentiable map for first-order
operators.
$$
\centerline{\epsfbox{tangmap15.1}}
$$

Diagram tangmap16. Differential of a differentiable map for second-order
operators.
$$
\centerline{\epsfbox{tangmap16.1}}
$$

% - - - - - - - - - - - - - - - - - - - - - - - - - - - - - - - - - - - - - - -
\filleject

Diagram tangmap17. Differential of a differentiable map for second-order vectors
and operators.
$$
\centerline{\epsfbox{tangmap17.1}}
$$

Diagram tangmap18. Induced map for a manifold map.
$$
\centerline{\epsfbox{tangmap18.1}}
$$

Diagram tangmap19. Diffeomorphism acting on a space of operators.
$$
\centerline{\epsfbox{tangmap19.1}}
$$

Diagram tangmap20. Transition maps for tangent space of a tangent space.
$$
\centerline{\epsfbox{tangmap20.1}}
$$

% - - - - - - - - - - - - - - - - - - - - - - - - - - - - - - - - - - - - - - -
\filleject

Diagram tangmap21. Tangent space metadefinition transition maps.
$$
\centerline{\epsfbox{tangmap21.1}}
$$

Diagram tangmap22. Tangent space metadefinition. Based on tangmap21.
$$
\centerline{\epsfbox{tangmap22.1}}
$$

Diagram tangmap23. Maps for the tangent bundle of a tangent bundle.
$$
\centerline{\epsfbox{tangmap23.1}}
$$

Diagram tangmap24. Standard atlas for a tangent bundle.
$$
\centerline{\epsfbox{tangmap24.1}}
$$

% - - - - - - - - - - - - - - - - - - - - - - - - - - - - - - - - - - - - - - -
\filleject

Diagram tangmap25. Tangent space metadefinition. Based on tangmap22.
$$
\centerline{\epsfbox{tangmap25.1}}
$$

Diagram tangmap26. Second-order tangent space definition. Based on tangmap25.
$$
\centerline{\epsfbox{tangmap26.1}}
$$

Diagram tangmap27. Invariance of tangent vector under a pseudogroup of
diffeomorphisms.
$$
\centerline{\epsfbox{tangmap27.1}}
$$

Diagram tangmap28. Transformation of second-order operators without a
connection. Based on tangmap19.
$$
\centerline{\epsfbox{tangmap28.1}}
$$

% - - - - - - - - - - - - - - - - - - - - - - - - - - - - - - - - - - - - - - -
\filleject

Diagram tangmap29. Mixed tensors on manifolds and Cartesian charts.
$$
\centerline{\epsfbox{tangmap29.1}}
$$

Diagram tangmap30. Maps and spaces for a continuous embedding of a topological
manifold.
$$
\centerline{\epsfbox{tangmap30.1}}
$$

Diagram tangmap31. Regular embedding of a submanifold of a topological manifold.
$$
\centerline{\epsfbox{tangmap31.1}}
$$

% - - - - - - - - - - - - - - - - - - - - - - - - - - - - - - - - - - - - - - -
\filleject

Diagram tangmap32. Charts and projection maps for tangent and covector bundles.
$$
\centerline{\epsfbox{tangmap32.1}}
$$

Diagram tangmap33. Composition of curves and real-valued functions on a
differentiable manifold.
$$
\centerline{\epsfbox{tangmap33.1}}
$$

Diagram tangmap34. Double differential of a map between two manifolds.
$$
\centerline{\epsfbox{tangmap34.1}}
$$

% - - - - - - - - - - - - - - - - - - - - - - - - - - - - - - - - - - - - - - -
\filleject

Diagram tangmap35. Covariant double differential of a map between two manifolds.
$$
\centerline{\epsfbox{tangmap35.1}}
$$

Diagram tangmap36. Covariant double differential of real-valued function on a
manifold.
$$
\centerline{\epsfbox{tangmap36.1}}
$$

Diagram tangmap37. Differentiability test for a vector field.
$$
\centerline{\epsfbox{tangmap37.1}}
$$

Diagram tangmap38. Differentiability test for a tensor field.
$$
\centerline{\epsfbox{tangmap38.1}}
$$

% - - - - - - - - - - - - - - - - - - - - - - - - - - - - - - - - - - - - - - -
\filleject

Diagram tangmap39. Drop function for the tangent bundle of a linear space.
$$
\centerline{\epsfbox{tangmap39.1}}
$$

Diagram tangmap40. Maps and spaces for a tangent vector $r$-frame bundle.
$$
\centerline{\epsfbox{tangmap40.1}}
$$

Diagram tangmap41. Maps and spaces for a tangent vector $r$-tuple bundle.
$$
\centerline{\epsfbox{tangmap41.1}}
$$

Diagram tangmap42. Pull-back differential of a map between two manifolds,
vector-valued.
$$
\centerline{\epsfbox{tangmap42.1}}
$$

% - - - - - - - - - - - - - - - - - - - - - - - - - - - - - - - - - - - - - - -
\filleject

Diagram tangmap43. Pull-back differential of a map between two manifolds,
real-valued.
$$
\centerline{\epsfbox{tangmap43.1}}
$$

Diagram tangmap44. Pointwise differential of a map between two manifolds.
$$
\centerline{\epsfbox{tangmap44.1}}
$$

Diagram tangmap45. Exterior derivative of differential forms on manifolds.
$$
\centerline{\epsfbox{tangmap45.1}}
$$

Diagram tangmap46. Exterior derivative of differential forms on manifolds,
without and with the ``short-cut''.
$$
\centerline{\epsfbox{tangmap46.1}}
$$

% - - - - - - - - - - - - - - - - - - - - - - - - - - - - - - - - - - - - - - -
\filleject

Diagram tangmap47. Differential forms on manifolds, without and with the
``short-cut''.
$$
\centerline{\epsfbox{tangmap47.1}}
$$

Diagram tangmap48. Differential forms on manifolds, without and with the
``short-cut''.
$$
\centerline{\epsfbox{tangmap48.1}}
$$

%==============================================================================
\secteject
\edef\SECTtopo{\the\pageno}\LinkName{topo}

Diagram topo1. $\tc_0$ and $\tc_1$ topological spaces.
$$
\centerline{\epsfbox{topo1.1}}
$$

Diagram topo2. Hausdorff ($\tc_2$) topological space.
$$
\centerline{\epsfbox{topo2.1}}
$$

Diagram topo3. $\tc_4$ topological space.
$$
\centerline{\epsfbox{topo3.1}}
$$

Diagram topo4. $\tc_4$ topological space.
$$
\centerline{\epsfbox{topo4.1}}
$$

Diagram topo5. Definition of connected set in a topological space.
$$
\centerline{\epsfbox{topo5.1}}
$$

Diagram topo6. Completely regular space. Existence of function in range~$[0,1]$.
$$
\centerline{\epsfbox{topo6.1}}
$$

% - - - - - - - - - - - - - - - - - - - - - - - - - - - - - - - - - - - - - - -
\filleject

Diagram topo7. DG level 1. Connectedness of point sets.
$$
\centerline{\epsfbox{topo7.1}}
$$

Diagram topo8. DG level 1. Define continuity of a function as preservation of
connectivity.
$$
\centerline{\epsfbox{topo8.1}}
$$

Diagram topo9. DG level 1. Define continuity of a function as inverse
preservation of disconnectivity.
$$
\centerline{\epsfbox{topo9.1}}
$$

Diagram topo10. DG level 1. \Neighbour/hoods around each point of two sets.
$$
\centerline{\epsfbox{topo10.1}}
$$

Diagram topo11. DG level 1. Union of \neighbour/hoods around points of two sets.
$$
\centerline{\epsfbox{topo11.1}}
$$

% - - - - - - - - - - - - - - - - - - - - - - - - - - - - - - - - - - - - - - -
\filleject

Diagram topo12. Topological limit point of a set.
$$
\centerline{\epsfbox{topo12.1}}
$$

Diagram topo13. Topological limit point of a set.
$$
\centerline{\epsfbox{topo13.1}}
$$

Diagram topo14. Topological interior, exterior and boundary of a set.
$$
\centerline{\epsfbox{topo14.1}}
$$

Diagram topo15. Open and closed portions of the boundary of a set.
$$
\centerline{\epsfbox{topo15.1}}
$$

% - - - - - - - - - - - - - - - - - - - - - - - - - - - - - - - - - - - - - - -
\filleject

Diagram topo16. The difference between weakly and strongly separated set-pairs.
$$
\centerline{\epsfbox{topo16.1}}
$$

Diagram topo17. The influence of topology strength on boundary thickness.
$$
\centerline{\epsfbox{topo17.1}}
$$

Diagram topo18. Relation between topology strength and
interior/boundary/exterior of sets.
$$
\centerline{\epsfbox{topo18.1}}
$$

% - - - - - - - - - - - - - - - - - - - - - - - - - - - - - - - - - - - - - - -
\filleject

Diagram topo19. Interior/boundary/exterior of sets including trivial and
discrete extremes.
$$
\centerline{\epsfbox{topo19.1}}
$$

Diagram topo20. Relations between interior, closure, boundary and exterior of a
set.
$$
\centerline{\epsfbox{topo20.1}}
$$

Diagram topo21. Inverse set-maps of interior, exterior and boundary for a
continuous function.
$$
\centerline{\epsfbox{topo21.1}}
$$

Diagram topo22. Least upper bound and greatest lower bound.
$$
\centerline{\epsfbox{topo22.1}}
$$

% - - - - - - - - - - - - - - - - - - - - - - - - - - - - - - - - - - - - - - -
\filleject

Diagram topo27. Projection of open set in Cartesian product of topological
spaces.
$$
\centerline{\epsfbox{topo27.1}}
$$

Diagram topo28. Topological space of class~$\tc_3$. $K$~closed, $x\notin K$,
$K\subseteq\Omega_1$, $x\in\Omega_2$, $\Omega_1\bcap\Omega_2=\emptyset$.
$$
\centerline{\epsfbox{topo28.1}}
$$

Diagram topo29. $\tc_5$ topological space. Weak separation of set-pairs
$(S_1,S_2)$ implies strong separation.
$$
\centerline{\epsfbox{topo29.1}}
$$

Diagram topo30. $\tc_6$ topological space.
$$
\centerline{\epsfbox{topo30.1}}
$$

Diagram topo31. Compare/contrast limit points, isolated points and
interior/boundary for~$\textop{iso}(X)=\emptyset$.
$$
\centerline{\epsfbox{topo31.1}}
$$

% - - - - - - - - - - - - - - - - - - - - - - - - - - - - - - - - - - - - - - -
\filleject

Diagram topo32. Compare and contrast limit points, isolated points and
interior/boundary/exterior.
$$
\centerline{\epsfbox{topo32.1}}
$$

Diagram topo33. Relations between interior and closure of a set.
$$
\centerline{\epsfbox{topo33.1}}
$$

Diagram topo35. A topological manifold chart for a torus.
$$
\centerline{\epsfbox{topo35.1}}
$$

Diagram topo39. Equivalent nested open/closed condition for a $\tc_3$ space.
$$
\centerline{\epsfbox{topo39.1}}
$$

% - - - - - - - - - - - - - - - - - - - - - - - - - - - - - - - - - - - - - - -
\filleject

Diagram topo40. Compare and constrast atlas-collages and identification spaces.
$$
\centerline{\epsfbox{topo40.1}}
$$

%==============================================================================
\secteject
\edef\SECTtopofin{\the\pageno}\LinkName{topofin}

Diagram topofin1. Non-discrete $T_1$ topology on a countably infinite set.
$$
\centerline{\epsfbox{topofin1.1}}
$$

Diagram topofin2. The set of all possible topologies on the set~$\{1,2\}$.
$$
\centerline{\epsfbox{topofin2.1}}
$$

Diagram topofin3. The set of unique topologies on the set~$\{1,2,3\}$.
$$
\centerline{\epsfbox{topofin3.1}}
$$

Diagram topofin4. Counterexamples for weakly and strongly connected function
conjectures.
$$
\centerline{\epsfbox{topofin4.1}}
$$

Diagram topofin5. Weakly connected function with non-$\tc_1$ target space which
is not continuous.
$$
\centerline{\epsfbox{topofin5.1}}
$$

Diagram topofin6. Two topologies on the set~$\{1,2,3\}$. Adapted from topofin3.
$$
\centerline{\epsfbox{topofin6.1}}
$$

% - - - - - - - - - - - - - - - - - - - - - - - - - - - - - - - - - - - - - - -
% \filleject

%==============================================================================
\secteject
\edef\SECTtopomap{\the\pageno}\LinkName{topomap}

Diagram topomap1. Re\parametris/ation of a rectifiable curve.
$$
\centerline{\epsfbox{topomap1.1}}
$$

Diagram topomap2. Re\parametris/ation of two curves:
$\gamma_1=\gamma_3\circ\beta_1$, $\gamma_2=\gamma_3\circ\beta_2$.
$$
\centerline{\epsfbox{topomap2.1}}
$$

Diagram topomap3. Effect of an atlas chart-transition map on an open set.
$$
\centerline{\epsfbox{topomap3.1}}
$$

Diagram topomap4. Relations between sets, topologies and atlases.
$$
\centerline{\epsfbox{topomap4.1}}
$$

% - - - - - - - - - - - - - - - - - - - - - - - - - - - - - - - - - - - - - - -
\filleject

Diagram topomap5. Continuity of composition of partially defined functions.
$$
\centerline{\epsfbox{topomap5.1}}
$$

Diagram topomap6. Re\parametris/ation of curve $\gamma_1$ to remove constant
stretches.
$$
\centerline{\epsfbox{topomap6.1}}
$$

Diagram topomap7. Equivalence of curves $\gamma_1$ and $\gamma_2$ via
re\parametris/ations $\beta_1$ and~$\beta_2$.
$$
\centerline{\epsfbox{topomap7.1}}
$$

Diagram topomap8. Re-reparametrisation of curves to use open-interval domains.
$$
\centerline{\epsfbox{topomap8.1}}
$$

% - - - - - - - - - - - - - - - - - - - - - - - - - - - - - - - - - - - - - - -
\filleject

Diagram topomap9. Method of proof of transitivity of the path-equivalence
relation.
$$
\centerline{\epsfbox{topomap9.1}}
$$

Diagram topomap10. Path-equivalence of two curves via a never-constant curve.
$$
\centerline{\epsfbox{topomap10.1}}
$$

%==============================================================================
\secteject
\edef\SECTtree{\the\pageno}\LinkName{tree}

Diagram tree1. Just testing\dots
$$
\centerline{\epsfbox{tree1.1}}
$$

Diagram tree2. A family tree of modules and algebras.
$$
\centerline{\epsfbox{tree2.1}}
$$

Diagram tree3. A family tree of groups, rings and fields. See also tree18 and
tree39.
$$
\centerline{\epsfbox{tree3.1}}
$$

% - - - - - - - - - - - - - - - - - - - - - - - - - - - - - - - - - - - - - - -
\filleject

Diagram tree4. A family tree of (compact) topological spaces.
$$
\centerline{\epsfbox{tree4.1}}
$$

Diagram tree5. Family tree of topological and Lie transformation groups.
$$
\centerline{\epsfbox{tree5.1}}
$$

Diagram tree6. Topological and Lie transformation groups. Rearangement and
augmentation of tree5.
$$
\centerline{\epsfbox{tree6.1}}
$$

Diagram tree7. Topological and Lie groups.
$$
\centerline{\epsfbox{tree7.1}}
$$

% - - - - - - - - - - - - - - - - - - - - - - - - - - - - - - - - - - - - - - -
\filleject

Diagram tree8. Compactness and separation classes. (Variant of diagram tree4.)
$$
\centerline{\epsfbox{tree8.1}}
$$

Diagram tree9. Connectivity and separation classes. (See tree14.)
$$
\centerline{\epsfbox{tree9.1}}
$$

Diagrams tree10 (left) and tree11 (right). Dependencies of chapters.
$$
\centerline{\epsfbox{tree10.1}\hss\epsfbox{tree11.1}}
$$

% - - - - - - - - - - - - - - - - - - - - - - - - - - - - - - - - - - - - - - -
\filleject

Diagrams tree12 (left) and tree16 (right). Reverse-time and forward-time arrows.
$$
\centerline{\epsfbox{tree12.1}\hss\epsfbox{tree16.1}}
$$

Diagram tree13. Topological \fibre/ bundles.
$$
\centerline{\epsfbox{tree13.1}}
$$

Diagram tree14. Connectivity classes and topological manifolds. (Based on
tree9.)
$$
\centerline{\epsfbox{tree14.1}}
$$

% - - - - - - - - - - - - - - - - - - - - - - - - - - - - - - - - - - - - - - -
\filleject

Diagram tree15. Differentiable \fibre/ bundles.
$$
\centerline{\epsfbox{tree15.1}}
$$

Diagram tree16. See tree12.

Diagram tree17. Topological \fibre/ bundles.
$$
\centerline{\epsfbox{tree17.1}}
$$

Diagram tree18. Semigroups, groups, rings and fields. Adapted from tree3.
$$
\centerline{\epsfbox{tree18.1}}
$$

% - - - - - - - - - - - - - - - - - - - - - - - - - - - - - - - - - - - - - - -
\filleject

Diagram tree19. Separation classes of topologies.
$$
\centerline{\epsfbox{tree19.1}}
$$

Diagram tree20. \Fibre/ bundles.
$$
\centerline{\epsfbox{tree20.1}}
$$

Diagram tree21. Functions and relations.
$$
\centerline{\epsfbox{tree21.1}}
$$

Diagram tree22. Mathematicians, physicists and reality.
$$
\centerline{\epsfbox{tree22.1}}
$$

Diagram tree23. Levels of structure on manifolds.
$$
\centerline{\epsfbox{tree23.1}}
$$

% - - - - - - - - - - - - - - - - - - - - - - - - - - - - - - - - - - - - - - -
\filleject

Diagram tree24. Family tree for \fibre/ bundles, tangent bundles and tensor
bundles.
$$
\centerline{\epsfbox{tree24.1}}
$$

Diagram tree25. Relations between tangent bundles, Lie groups and differentiable
\fibre/ bundles.
$$
\centerline{\epsfbox{tree25.1}}
$$

Diagram tree26. Family tree for sets of matrices.
$$
\centerline{\epsfbox{tree26.1}}
$$

Diagram tree27. Family tree for relations, functions and order relations.
$$
\centerline{\epsfbox{tree27.1}}
$$

% - - - - - - - - - - - - - - - - - - - - - - - - - - - - - - - - - - - - - - -
\filleject

Diagram tree28. Family tree for DG topics. Based on tree16.
$$
\centerline{\epsfbox{tree28.1}}
$$

Diagram tree29. Differentiability properties for real functions of multiple
variables. (Probably wrong!)
$$
\centerline{\epsfbox{tree29.1}}
$$

Diagram tree30. Differentiability properties for real functions of multiple
variables.
$$
\centerline{\epsfbox{tree30.1}}
$$

Diagram tree31. Classification of differential equations.
$$
\centerline{\epsfbox{tree31.1}}
$$

% - - - - - - - - - - - - - - - - - - - - - - - - - - - - - - - - - - - - - - -
\filleject

Diagram tree32. Family tree of topology separation classes.
$$
\centerline{\epsfbox{tree32.1}}
$$

Diagram tree33. Conjectural family tree for certain primates.
$$
\centerline{\epsfbox{tree33.1}}
$$

Diagram tree34. Family tree of linear algebra definitions.
$$
\centerline{\epsfbox{tree34.1}}
$$

% - - - - - - - - - - - - - - - - - - - - - - - - - - - - - - - - - - - - - - -
\filleject

Diagram tree35. Relations between \fibre/ bundles and tensor bundles.
$$
\centerline{\epsfbox{tree35.1}}
$$

Diagram tree36. Tree of definitions of classes of subsets of a power set.
$$
\centerline{\epsfbox{tree36.1}}
$$

Diagram tree37. Tree of definitions of classes and $\sigma$-classes of subsets
of a power set.
$$
\centerline{\epsfbox{tree37.1}}
$$

Diagram tree38. Family tree for relations, functions and order relations. Based
on tree27.
$$
\centerline{\epsfbox{tree38.1}}
$$

% - - - - - - - - - - - - - - - - - - - - - - - - - - - - - - - - - - - - - - -
\filleject

Diagram tree39. Semigroups, groups, rings and fields. Adapted from tree18.
$$
\centerline{\epsfbox{tree39.1}}
$$

Diagram tree40. Semigroups, groups, transformation groups, etc. (From tree39.)
$$
\centerline{\epsfbox{tree40.1}}
$$

Diagram tree41. A family tree of affine spaces.
$$
\centerline{\epsfbox{tree41.1}}
$$

% - - - - - - - - - - - - - - - - - - - - - - - - - - - - - - - - - - - - - - -
\filleject

Diagram tree42. Family tree for order relations. Based on tree38.
$$
\centerline{\epsfbox{tree42.1}}
$$

Diagram tree43. Classes of bounds for a subset $A$ of a partially orderered
set~$X$.
$$
\centerline{\epsfbox{tree43.1}}
$$

Diagram tree44. Family tree for order relations, including various
well-orderings.
$$
\centerline{\epsfbox{tree44.1}}
$$

Diagram tree45. Family tree for order relations, including global and local
well-orderings.
$$
\centerline{\epsfbox{tree45.1}}
$$

% - - - - - - - - - - - - - - - - - - - - - - - - - - - - - - - - - - - - - - -
\filleject

Diagram tree46. Family tree for order relations, including global and local
well-orderings.
$$
\centerline{\epsfbox{tree46.1}}
$$

Diagram tree47. Family tree for semigroups, groups, rings and fields. Based on
tree39.
$$
\centerline{\epsfbox{tree47.1}}
$$

Diagram tree48. Family tree for non-topological fibrations and \fibre/ bundles.
$$
\centerline{\epsfbox{tree48.1}}
$$

Diagram tree49. Cycle of constructions of tangent bundles, \fibre/ atlases and
structure groups.
$$
\centerline{\epsfbox{tree49.1}}
$$

% - - - - - - - - - - - - - - - - - - - - - - - - - - - - - - - - - - - - - - -
\filleject

Diagram tree50. Topology separation classes, with and without~AC. (See also
tree67.)
$$
\centerline{\epsfbox{tree50.1}}
$$

Diagram tree51. Family tree for Riemannian manifolds.
$$
\centerline{\epsfbox{tree51.1}}
$$

Diagram tree52. Family tree for locally connected and pathwise connected spaces.
$$
\centerline{\epsfbox{tree52.1}}
$$

Diagram tree53. Relations between definitions of ``infinity''.
$$
\centerline{\epsfbox{tree53.1}}
$$

% - - - - - - - - - - - - - - - - - - - - - - - - - - - - - - - - - - - - - - -
\filleject

Diagram tree54. Relations between logical language, logical propositions and
mathematical systems.
$$
\centerline{\epsfbox{tree54.1}}
$$

Diagram tree55. Relations between propositions and formulas, and language and
interpretation.
$$
\centerline{\epsfbox{tree55.1}}
$$

Diagram tree56. Relations between propositions, logical expressions and
knowledge sets.
$$
\centerline{\epsfbox{tree56.1}}
$$

Diagram tree57. Family tree for order relations, including various
well-orderings. (Based on tree44.mp.)
$$
\centerline{\epsfbox{tree57.1}}
$$

% - - - - - - - - - - - - - - - - - - - - - - - - - - - - - - - - - - - - - - -
\filleject

Diagram tree58. Deriving pseudo-Riemannian manifolds from Riemannian manifolds
and Minkowski spaces.
$$
\centerline{\epsfbox{tree58.1}}
$$

Diagram tree59. Family tree for Riemannian manifolds. Based on tree51.
$$
\centerline{\epsfbox{tree59.1}}
$$

Diagram tree60. Family tree for Riemannian manifolds. Based on tree59.
$$
\centerline{\epsfbox{tree60.1}}
$$

Diagram tree61. Family tree for Riemannian manifolds. Based on tree60.
$$
\centerline{\epsfbox{tree61.1}}
$$

% - - - - - - - - - - - - - - - - - - - - - - - - - - - - - - - - - - - - - - -
\filleject

Diagram tree62. Classes of logic systems.
$$
\centerline{\epsfbox{tree62.1}}
$$

Diagram tree63. Layer diagram for mathematics language and semantics.
$$
\centerline{\epsfbox{tree63.1}}
$$

Diagram tree64. Family tree for Riemannian manifolds. Based on tree60.
$$
\centerline{\epsfbox{tree64.1}}
$$

Diagram tree65. Family tree for topological linear spaces.
$$
\centerline{\epsfbox{tree65.1}}
$$

% - - - - - - - - - - - - - - - - - - - - - - - - - - - - - - - - - - - - - - -
\filleject

Diagram tree66. Classes of finite and infinite sets. (See also tree69.)
$$
\centerline{\epsfbox{tree66.1}}
$$

Diagram tree67. Topology separation classes, without and with~AC. (Based on
tree50.)
$$
\centerline{\epsfbox{tree67.1}}
$$

Diagram tree68. Relations between separable and second/first countable
topological spaces.
$$
\centerline{\epsfbox{tree68.1}}
$$

% - - - - - - - - - - - - - - - - - - - - - - - - - - - - - - - - - - - - - - -
\filleject

Diagram tree69. Classes of finite and infinite sets. (Based on tree66.)
$$
\centerline{\epsfbox{tree69.1}}
$$

Diagram tree70. Implications and impossible implications between some ZF choice
axioms.
$$
\centerline{\epsfbox{tree70.1}}
$$

Diagram tree71. Family tree for classes of $\omega$-infinite, countable and
uncountable sets.
$$
\centerline{\epsfbox{tree71.1}}
$$

% - - - - - - - - - - - - - - - - - - - - - - - - - - - - - - - - - - - - - - -
\filleject

Diagram tree72. Classes of finite and infinite sets. (Based on tree69.)
$$
\centerline{\epsfbox{tree72.1}}
$$

Diagram tree73. Relations between countable compactness and some other
properties.
$$
\centerline{\epsfbox{tree73.1}}
$$

Diagram tree74. Relations between limit-point compactness, countable compactness
and other properties.
$$
\centerline{\epsfbox{tree74.1}}
$$

Diagram tree75. Relations between limit-point compactness, countable compactness
and other properties.
$$
\centerline{\epsfbox{tree75.1}}
$$

% - - - - - - - - - - - - - - - - - - - - - - - - - - - - - - - - - - - - - - -
\filleject

Diagram tree76. Relations between limit-point, countable and sequential
compactness.
$$
\centerline{\epsfbox{tree76.1}}
$$

Diagram tree77. Lipschitz class relations.
$$
\centerline{\epsfbox{tree77.1}}
$$

Diagram tree78. Family tree for rings.
$$
\centerline{\epsfbox{tree78.1}}
$$

% - - - - - - - - - - - - - - - - - - - - - - - - - - - - - - - - - - - - - - -
\filleject

Diagram tree79. Family tree for locally Cartesian spaces.
$$
\centerline{\epsfbox{tree79.1}}
$$

Diagram tree80. Family tree for locally Cartesian spaces with atlases.
$$
\centerline{\epsfbox{tree80.1}}
$$

Diagram tree81. Family tree for covers and relations.
$$
\centerline{\epsfbox{tree81.1}}
$$

Diagram tree82. The relation of mathematics foundations to quality of life.
$$
\centerline{\epsfbox{tree82.1}}
$$

Diagram tree83. Basic family tree for functions.
$$
\centerline{\epsfbox{tree83.1}}
$$

% - - - - - - - - - - - - - - - - - - - - - - - - - - - - - - - - - - - - - - -
\filleject

Diagram tree84. Family tree for semigroups, groups, rings and integral systems.
$$
\centerline{\epsfbox{tree84.1}}
$$

Diagram tree85. Semigroups, groups, transformation groups, etc. (From tree40.)
$$
\centerline{\epsfbox{tree85.1}}
$$

Diagram tree86. Existence of non-decreasing continuous surjections between
real-number intervals.
$$
\centerline{\epsfbox{tree86.1}}
$$

Diagram tree87. Lipschitz class relations.
$$
\centerline{\epsfbox{tree87.1}}
$$

% - - - - - - - - - - - - - - - - - - - - - - - - - - - - - - - - - - - - - - -
\filleject

Diagram tree88. Differentiability classes for maps between Cartesian spaces.
$$
\centerline{\epsfbox{tree88.1}}
$$

Diagram tree89. Family tree of differentiable \fibre/ bundles for connection
styles.
$$
\centerline{\epsfbox{tree89.1}}
$$

%==============================================================================
\secteject
\edef\SECTvec{\the\pageno}\LinkName{vec}

Diagram vec1. Equivalent tensor products of vector pairs.
$$
\centerline{\epsfbox{vec1.1}}
$$

Diagram vec2. \Visualis/ation of tangent covector in flat space.
$$
\centerline{\epsfbox{vec2.1}}
$$

Diagram vec3. \Visualis/ation of tangent vectors and covectors in flat space.
$$
\centerline{\epsfbox{vec3.1}}
$$

Diagram vec4. Affine space transitivity.
$$
\centerline{\epsfbox{vec4.1}}
$$

Diagram vec5. Commutativity of linear space addition.
$$
\centerline{\epsfbox{vec5.1}}
$$

% - - - - - - - - - - - - - - - - - - - - - - - - - - - - - - - - - - - - - - -
\filleject

Diagram vec6. Commutativity of linear space addition.
$$
\centerline{\epsfbox{vec6.1}}
$$

Diagram vec7. Parallel motion under a connection.
$$
\centerline{\epsfbox{vec7.1}}
$$

Diagram vec8. Parallel motion under a connection with group invariance.
$$
\centerline{\epsfbox{vec8.1}}
$$

Diagram vec9. Parallel translation of vector with respect to a chart.
$$
\centerline{\epsfbox{vec9.1}}
$$

Diagram vec10. Second differential of a path.
$$
\centerline{\epsfbox{vec10.1}}
$$

% - - - - - - - - - - - - - - - - - - - - - - - - - - - - - - - - - - - - - - -
\filleject

Diagram vec11. Parallel transport of tangent vectors.
$$
\centerline{\epsfbox{vec11.1}}
$$

Diagram vec12. The difference between left and right conjugation for
$\gp{GL}(2)$. Based on map85.
$$
\centerline{\epsfbox{vec12.1}}
$$

Diagram vec13. Transformation of tangent vector of curve under diffeomorphism.
$$
\centerline{\epsfbox{vec13.1}}
$$

Diagram vec14. Rate of change of a vector in flat space.
$$
\centerline{\epsfbox{vec14.1}}
$$

% - - - - - - - - - - - - - - - - - - - - - - - - - - - - - - - - - - - - - - -
\filleject

Diagram vec15. Transformation of second-order tangent vector under
diffeomorphism.
$$
\centerline{\epsfbox{vec15.1}}
$$

Diagram vec16. Two equivalent vector-pair tensor products. (Based on diagram
vec1.)
$$
\centerline{\epsfbox{vec16.1}}
$$

Diagram vec17. Effect of rank-1 tensor on linear functionals. (Based on diagram
vec2.)
$$
\centerline{\epsfbox{vec17.1}}
$$

Diagram vec18. Equivalent vector-pair tensor products. (Based on diagram
vec16.)
$$
\centerline{\epsfbox{vec18.1}}
$$

% - - - - - - - - - - - - - - - - - - - - - - - - - - - - - - - - - - - - - - -
\filleject

Diagram vec19. Level curves of dual space basis vector for varying primal basis.
$$
\centerline{\epsfbox{vec19.1}}
$$

Diagram vec20. \Parametris/ed lines in Cartesian space.
$$
\centerline{\epsfbox{vec20.1}}
$$

Diagram vec21. \Parametris/ed lines in Cartesian space. Addition operation.
$$
\centerline{\epsfbox{vec21.1}}
$$

Diagram vec22. Transformation of linear-trajectory velocity of curve under
diffeomorphism.
$$
\centerline{\epsfbox{vec22.1}}
$$

% - - - - - - - - - - - - - - - - - - - - - - - - - - - - - - - - - - - - - - -
\filleject

Diagram vec23. Invariance of affine space transitivity under affine
transformations. (Work in progress.)
$$
\centerline{\epsfbox{vec23.1}}
$$

Diagram vec24. Span and independence of vectors in a linear space.
$$
\centerline{\epsfbox{vec24.1}}
$$

Diagram vec25. Components of second-level tangent vector in point/velocity form.
$$
\centerline{\epsfbox{vec25.1}}
$$

Diagram vec26. Components of second-level tangent vector in parametrised line
form.
$$
\centerline{\epsfbox{vec26.1}}
$$

% - - - - - - - - - - - - - - - - - - - - - - - - - - - - - - - - - - - - - - -
\filleject

Diagram vec27. Commutativity of derivatives of a $C^2$ family of curves in a
manifold.
$$
\centerline{\epsfbox{vec27.1}}
$$

Diagram vec28. Proof that subspaces of finite-dimensional spaces are
finite-dimensional.
$$
\centerline{\epsfbox{vec28.1}}
$$

Diagram vec29. Computation of the Levi-Civita connection from the metric
function.
$$
\centerline{\epsfbox{vec29.1}}
$$

Diagram vec30. Computation of negative of the Levi-Civita connection from the
metric function.
$$
\centerline{\epsfbox{vec30.1}}
$$

% - - - - - - - - - - - - - - - - - - - - - - - - - - - - - - - - - - - - - - -
\filleject

Diagram vec31. The difference between left and right conjugation for
$\gp{GL}(2)$. Mirror image of vec12.
$$
\centerline{\epsfbox{vec31.1}}
$$

%==============================================================================
\secteject
\edef\SECTthreeD{\the\pageno}\LinkName{threeD}

Diagram 3d1. 3-d axes.
$$
\centerline{\epsfbox{3d1.1}}
$$

Diagram 3d2. Basic axes in $\reals^3$ with grid on the X-Y plane.
$$
\centerline{\epsfbox{3d2.1}}
$$

Diagram 3d3. Another perspective for diagram 3d2.
$$
\centerline{\epsfbox{3d3.1}}
$$

Diagram 3d4. Basis vector for constructing a tensor product.
$$
\centerline{\epsfbox{3d4.1}}
$$

Diagram 3d5. The indicator (characteristic) function of a single point.
Long-distance shot.
$$
\centerline{\epsfbox{3d5.1}}
$$

% - - - - - - - - - - - - - - - - - - - - - - - - - - - - - - - - - - - - - - -
\filleject

Diagram 3d6. Transparent wire-frame sphere.
$$
\centerline{\epsfbox{3d6.1}}
$$

Diagram 3d7. Transparent wire-frame sphere with reflection.
$$
\centerline{\epsfbox{3d7.1}}
$$

% - - - - - - - - - - - - - - - - - - - - - - - - - - - - - - - - - - - - - - -
\filleject

Diagram 3d8. A wire-frame half-sphere.
$$
\centerline{\epsfbox{3d8.1}}
$$

Diagram 3d9. A wire-frame sphere with line hiding.
$$
\centerline{\epsfbox{3d9.1}}
$$

Diagram 3d10. A 2-sphere demonstrating path-dependent parallel transport.
$$
\centerline{\epsfbox{3d10.1}}
$$

% - - - - - - - - - - - - - - - - - - - - - - - - - - - - - - - - - - - - - - -
\filleject

Diagram 3d11. Same as Diagram~3d9, but tilted and with less longitudes.
$$
\centerline{\epsfbox{3d11.1}}
$$

Diagram 3d12. Free linear space construction: $e_u=\chi_{\{u\}}$ for $\alpha=2$,
$V_1=V_2=K=\reals$, $u=(3,4)$.
$$
\centerline{\epsfbox{3d12.1}}
$$

Diagram 3d13. Free linear space construction: $e_u+e_v+e_w$ for $\alpha=2$,
$V_1=V_2=K=\reals$.
$$
\centerline{\epsfbox{3d13.1}}
$$

Diagram 3d14. Element
$\chi_{\{(3,3)\}}+2\chi_{\{(1,-4)\}}+(-3/2)\chi_{\{(-4,-1)\}}$ in free
linear space on $S=\reals^2$ over~$K=\reals$.
$$
\centerline{\epsfbox{3d14.1}}
$$

Diagram 3d15. The indicator (characteristic) function of a single point.
Close-up of 3d5.
$$
\centerline{\epsfbox{3d15.1}}
$$

% - - - - - - - - - - - - - - - - - - - - - - - - - - - - - - - - - - - - - - -
\filleject

Diagram 3d16. Path-dependent parallelism for triangular region with
$\theta=30^\circ$ edge.
$$
\centerline{\epsfbox{3d16.1}}
$$

Diagram 3d17. Double perspective version of 3d10.
$$
\centerline{\epsfbox{3d17.1}}
$$

Diagram 3d18. Sphere $S^2$ orbits under rotation around axis.
$$
\centerline{\epsfbox{3d18.1}}
$$

% - - - - - - - - - - - - - - - - - - - - - - - - - - - - - - - - - - - - - - -
\filleject

Diagram 3d19. Definition of multlinear map.
$$
\centerline{\epsfbox{3d19.1}}
$$

Diagram 3d20. Stokes theorem for a cube.
$$
\centerline{\epsfbox{3d20.1}}
$$

Diagram 3d21. Kronecker delta function on the integers.
$$
\centerline{\epsfbox{3d21.1}}
$$

% - - - - - - - - - - - - - - - - - - - - - - - - - - - - - - - - - - - - - - -
\filleject

Diagram 3d22. Stokes theorem for a cube. Based on diagram 3d20.
$$
\centerline{\epsfbox{3d22.1}}
$$

Diagram 3d23. A bucket.
$$
\centerline{\epsfbox{3d23.1}}
$$

Diagram 3d24. Dimensions of King Khufu's burial chamber in about 2500\thinspace
BC. 1\thinspace cubit = 0.523\thinspace m.
$$
\centerline{\epsfbox{3d24.1}}
$$

Diagram 3d25. Tangent line vectors in a Cartesian space. Space-time version of
diagram vec20.
$$
\centerline{\epsfbox{3d25.1}}
$$

% - - - - - - - - - - - - - - - - - - - - - - - - - - - - - - - - - - - - - - -
\filleject

Diagram 3d26. Tangent line vectors in a Cartesian space. Time-space version of
diagram 3d25.
$$
\centerline{\epsfbox{3d26.1}}
$$

Diagram 3d27. A slightly \colour/ful sphere $S^2\subseteq\reals^3$ in
perspective.
$$
\centerline{\epsfbox{3d27.1}}
$$

Diagram 3d28. Sphere $S^2\subseteq\reals^3$ without and with local charts.
$$
\centerline{\epsfbox{3d28.1}}
$$

% - - - - - - - - - - - - - - - - - - - - - - - - - - - - - - - - - - - - - - -
\filleject

Diagram 3d29. Torsion of parallel transport in flat three-dimensional space.
$$
\centerline{\epsfbox{3d29.1}}
$$

Diagram 3d30. Infinitesimal transformations of 2-sphere generated by Lie algebra
elements.
$$
\centerline{\epsfbox{3d30.1}}
$$

Diagram 3d31. Repeating pattern to be used for wild curve in $\reals^3$.
$$
\centerline{\epsfbox{3d31.1}}
$$

Diagram 3d32. Repeating pattern to be used for wild curve in $\reals^3$.
Different to 3d31.
$$
\centerline{\epsfbox{3d32.1}}
$$

% - - - - - - - - - - - - - - - - - - - - - - - - - - - - - - - - - - - - - - -
\filleject

Diagram 3d33. Wild knot in $\reals^3$.
$$
\centerline{\epsfbox{3d33.1}}
$$

Diagram 3d34. Continuous deformation of repeating knot pattern in 3d31.
$$
\centerline{\epsfbox{3d34.1}}
$$

Diagram 3d35. Sphere $S^2$ orbits under rotation around axis.
$$
\centerline{\epsfbox{3d35.1}}
$$

% - - - - - - - - - - - - - - - - - - - - - - - - - - - - - - - - - - - - - - -
\filleject

Diagram 3d36. Lift map from $\reals$ to $\reals^3$ through a
point~$p\in\reals^3$.
$$
\centerline{\epsfbox{3d36.1}}
$$

Diagram 3d37. Interpretation of the exterior derivative.
$$
\centerline{\epsfbox{3d37.1}}
$$

% - - - - - - - - - - - - - - - - - - - - - - - - - - - - - - - - - - - - - - -
% \filleject

\hbox to\hsize{%
\vtop{%
\halign{#\hfil&\space\hfil#\cr
\multispan2 \hfil index\hfil\cr
\noalign{\vskip3pt\hrule\vskip3pt}
arrow&\LinkHref{arrow}{\SECTarrow}\cr
axiom&\LinkHref{axiom}{\SECTaxiom}\cr
box&\LinkHref{box}{\SECTbox}\cr
calc&\LinkHref{calc}{\SECTcalc}\cr
chart&\LinkHref{chart}{\SECTchart}\cr
class&\LinkHref{class}{\SECTclass}\cr
connmap&\LinkHref{connmap}{\SECTconnmap}\cr
fibdiag&\LinkHref{fibdiag}{\SECTfibdiag}\cr
fibmap&\LinkHref{fibmap}{\SECTfibmap}\cr
fibre&\LinkHref{fibre}{\SECTfibre}\cr
field&\LinkHref{field}{\SECTfield}\cr
fn&\LinkHref{fn}{\SECTfn}\cr
fnmap&\LinkHref{fnmap}{\SECTfnmap}\cr
fn2d&\LinkHref{fnTWOd}{\SECTfnTWOd}\cr
grid&\LinkHref{grid}{\SECTgrid}\cr
int&\LinkHref{int}{\SECTint}\cr
ire&\LinkHref{ire}{\SECTire}\cr
\noalign{\vskip3pt\hrule\vskip3pt}
} % End of \halign.
} % End of \vbox.
\hskip6mm
\vtop{%
\halign{#\hfil&\space\hfil#\cr
\multispan2 \hfil index\hfil\cr
\noalign{\vskip3pt\hrule\vskip3pt}
linmap&\LinkHref{linmap}{\SECTlinmap}\cr
log&\LinkHref{log}{\SECTlog}\cr
logmap&\LinkHref{logmap}{\SECTlogmap}\cr
map&\LinkHref{map}{\SECTmap}\cr
mat&\LinkHref{mat}{\SECTmat}\cr
metric&\LinkHref{metric}{\SECTmetric}\cr
misc&\LinkHref{misc}{\SECTmisc}\cr
prodmap&\LinkHref{prodmap}{\SECTprodmap}\cr
real&\LinkHref{real}{\SECTreal}\cr
tangmap&\LinkHref{tangmap}{\SECTtangmap}\cr
topo&\LinkHref{topo}{\SECTtopo}\cr
topofin&\LinkHref{topofin}{\SECTtopofin}\cr
topomap&\LinkHref{topomap}{\SECTtopomap}\cr
tree&\LinkHref{tree}{\SECTtree}\cr
vec&\LinkHref{vec}{\SECTvec}\cr
3d&\LinkHref{threeD}{\SECTthreeD}\cr
\noalign{\vskip3pt\hrule\vskip3pt}
} % End of \halign.
} % End of \vbox.
\hss} % End of \hbox.

\bye
